%!TEX root = ../template.tex
%%%%%%%%%%%%%%%%%%%%%%%%%%%%%%%%%%%%%%%%%%%%%%%%%%%%%%%%%%%%%%%%%%%
%% chapter1.tex
%% NOVA thesis document file
%%
%% Chapter with introduciton
%%%%%%%%%%%%%%%%%%%%%%%%%%%%%%%%%%%%%%%%%%%%%%%%%%%%%%%%%%%%%%%%%%%
\newcommand{\novathesis}{\emph{novathesis}}
\newcommand{\novathesisclass}{\texttt{novathesis.cls}}



\chapter{Background and Motivation}%
\label{cha:bg_and_motivation}

\section{Context}%
\label{sec:context}

The idea behind this thesis was born in 2015, at NGNS-IS (a Portuguese
tech startup). At the time, the company's flagship product was the
Forest Fire Finder (\gls{FFF}). The \gls{FFF} was a forest fire
detection system, capable of mostly autonomous and automatic operation.
The system was the first application of Differential Optical Absorption
Spectroscopy~\gls{DOAS} for fire detection, and for that it was patented
in 2007 (see~\cite{Vieira2007, Application2008}). The \gls{FFF} is a
remote sensing device that scans the horizon for the presence of a smoke
column, sequentially performing a chemical analysis of each azimuth,
using the Sun as a light source for its spectroscopic
operations~\cite{ValentedeAlmeida2017}.

The \gls{FFF} was deployed in several "habitats", both nationally
(Parque Nacional da Peneda-Gerês and Ourém) and internationally (Spain
and Brazil). One of the company's clients at the time was interested in
a pollution monitoring solution, and asked if the spectroscopic system
would be capable of performing such a task. The challenge resonated
through the company's structure and the idea that created this thesis
was born. The team then started reading about the concept of Air
Pollution~\gls{AP} and how both populations and entities were concerned
about it. It became clear that, while there were already several methods
to measure \gls{AP}, there was a clear market drive for the development
of a system that could leverage the large area capabilities of a
\gls{DOAS} device while being able to provide a more spatially resolved
"picture" of the atmospheric status. With this in mind, the company
managed to have the investigation financed through a PT2020 funding
opportunity. This achievement was a clear validation of the project's
goals and of the need there was for a system with the proposed
capabilities. It was, however, not enough. \gls{FFF} was a very good
starting point, but there was still a lot of continuous research work
needed before any of the goals that had been set were achieved. This led
to the publication of this PhD project, in a tripartite consortium
between FCT-NOVA, NGNS-IS and the Portuguese Foundation for Science and
Technology. Its main goal was to develop an atmospheric monitoring
system prototype that would be able to spectroscopically map pollutant
concentrations in a two-dimensional way.

\section{The Problem}%
\label{sec:the_problem}

The first step in tackling the development of the proposed system was to
understand the problem it should be dealing with. Air Pollution
(\gls{AP}) is one of the most present concerns of people around the
industrialized modern world. In Europe, it is perceived to be the second
most important threat to the environment. The first is climate change,
which is great part caused by \gls{AP}. Scientists in many countries
have established it as a major cause for premature death, disease onset
and hospital visits for some decades now. Regulatory bodies of many
countries have been gathered to put some legislative pressure on
industries and on society itself, in order to produce a decrease in the
amount of Air Pollution to which people are exposed. These policies and
measures have had a dramatic influence in air quality, which is very
significantly improved throughout the years. In spite of this, official
reports continue to highlight the importance of keeping a vigilant eye
towards \gls{AP} proliferation and its possible undiscovered health
effects. A better introduction to the subject of \gls{AP} is produced in
Section~\ref{sec:problem_introduction}.

Our spectacular progress as a species in the last few centuries has had
some unforeseen adverse consequences. Mitigating them is not only a
responsibility, but also a necessity. In some regards, as is the
\gls{AP} case, this mitigation is only achievable with intelligent and
effective action. This of course demands that we understand, trace and
measure it in as many ways as possible. This project aimed to create
just that: another way in which to measure and map the behavior of
certain pollutants.

% \section{Objectives}%
% \label{sec:objectives}

% From the beginning of the project, the main objective has always been to
% design and develop a miniaturized spectroscopic environmental analysis
% device. The system would have to be small and portable enough to adapt
% to a drone, but should be able to function if adapted to any other
% surface, such as a car or even a fixed station. With time, however, the
% goal changed somewhat. The miniaturization and drone-adaptation were
% kept, but there was now the need to be able to map the pollutant
% concentration along a geographic region through the use of tomographic
% reconstruction algorithms. This meant that, while the device itself
% would be perfectly capable of operating without a drone, the tomographic
% capability would be lost, unless a prohibitively large number of
% spectroscopic systems were deployed in that region.

% This evolution in the scope of the project meant that the overarching
% goal of the project would now be to theorize and design a
% two-dimensional mapping tool for trace atmospheric pollutants such as
% NO$_x$ and SO$_x$, using passive DOAS as the measurement technique. In
% addition to this, it should be mentioned that the system must be small
% and portable; use tomographic reconstruction (and acquisition)
% algorithms to map the region; and that it should use only one collection
% point (to reduce costs and instrumental complexity). It was from this
% idea that we derived the research questions that we present in
% Section~\ref{sec:research_question}. 

\section{Open Questions}%
\label{sec:open_questions}

The literature on DOAS tomography is sparse. Although the theory behind
gas tomography is sound and well proved, the usage of this kind of
systems demands a very high level of technical proficiency, which is
chiefly found among scientists and researchers. Even then, given the
fact that tomographic systems are also very expensive, the number of
applications of this technology has been remarkably low, which results
in a series of open questions, which this project and this thesis aim to
exploit. 

Almost all the tomographic DOAS systems that were developed up until
this day have been deployed as active DOAS ground stations that targeted
a relatively small number of reflectors and as a result got a small
number of light paths in their measurements (see, for
instance,~\cite{Pundt2005}). These were unmovable devices that could in
fact operate autonomously, but were only able to map a fixed region's
trace gas profile. There have been, of course, some attempts at running
airborne tomographic DOAS experiments, but we could not find a single
one of these that would be able to run with no external interference. In
volcanology, passive tomographic DOAS is commonly used to sample
SO\textsubscript{2} profiles above the crater
(see~\cite{Johansson2009}). This kind of system also uses fixed ground
stations to gather the tomographic projections of a volcanic plume.
Although ingenious, their setup is also plagued by the problem of
insufficient information. It is very difficult to obtain a large
projection number using fixed ground stations, and this is turn leads to
poor image reconstructions. In Chapter~\ref{cha:literature_review}, I
present a more in depth review of the various applications the technique
has seen throughout the years.

It seems, as consequence of the above, that there are nowadays two large
open questions in the development of tomographic DOAS systems. On one
hand, current tomographic DOAS assemblies work with low projection
numbers, a limiting factor in the amount of detail the technique is able
to reproduce; on the other hand, the developers of these devices have
focused on the construction of unmovable ground stations, which
complicates their proliferation and probably play a role on the fact
that these systems are not at all widespread, in spite of their obvious
interesting capabilities. The main motivation for this thesis is thus to
explore these two large open questions, by developing a highly mobile
tomographic DOAS system, able to collect a higher number of tomographic
projections by being assembled onto an Unmanned Aerial Vehicle
(\gls{UAV}), which should also give the system the ability to be
autonomously deployed.

\section{Document Structure}%
\label{sec:document_structure}

This document is structured as follows: after this small introduction, I
dwell on the research questions and how they were derived
(Chapter~\ref{cha:research_question}); in
Chapter~\ref{cha:literature_review} I present comprehensive literature
review, conducted as a final requirement for one of the courses I
enrolled in during the PhD program; finally, in Chapter~\ref{cha:methodology}, I
present the research methodology that was employed for the development
of this thesis, and during the course of my work.


