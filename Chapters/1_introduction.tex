%!TEX root = ../thesis_rui_almeida.tex
%%%%%%%%%%%%%%%%%%%%%%%%%%%%%%%%%%%%%%%%%%%%%%%%%%%%%%%%%%%%%%%%%%%
%% chapter1.tex
%% Rui V. Almeida's thesis file
%%
%% Chapter with introduciton
%%%%%%%%%%%%%%%%%%%%%%%%%%%%%%%%%%%%%%%%%%%%%%%%%%%%%%%%%%%%%%%%%%%
\chapter{Introduction}%
\label{cha:introduction}

\section{Background, Motivation and Starting Points}%
\label{sec:background_and_motivation}

\subsection{Introduction}%
\label{sub:introduction}

This thesis describes the work that I have done in the past 4 years on
the design and development of a miniaturized system for atmospheric
monitoring based on optical spectroscopy. The project itself was the
major part of the \gls{ATMOS}, an initiative that was
contemplated with European funding through a \gls{PT2020} initiative and
came as a response to the growing weight that \gls{AP} has in the whole
Western world.

The potential impact of \gls{AP} on human health is amply documented.
Numerous papers have, for decades, established many links between air
quality and several common ailments like respiratory syndromes and
cardiovascular diseases. Similar connections have also been found
regarding the probability of gestational malformations and several types
of cancer. On a different level, and of perhaps less immediate concern,
are the effects that have been observed on ecosystems. Many times these
effects are difficult to predict (and timely mitigate) and in some cases
have been known to interfere with people's livelihood. In time, and if
not addressed, these interferences will certainly hinder economies and
limit the quality of life of populations globally. The severity of this
problem makes it clear that we need to tackle it intelligently, and this
approach requires that we can measure, trace and track \gls{AP}
effectively, which beckons engineers and scientists to create more
technology for this specific purpose.

Answering this call, with this work I have tried to create a reply to
the question of whether it would be possible to develop a
two-dimensional pollutant mapping tool, small enough to be fitted onto a
\gls{UAV}, which came to be a tomographically enabled design. To this
end, I have developed a simulation platform that computationally proves
the method's feasibility and confirmed through experiments that the
hypothesis on which the solution is based, regarding the use of
sequentially measured scattered sunlight as analogous to an artificial
light source is valid.

\subsection{Context}%
\label{sub:context}

The idea behind this thesis was born in 2015, at NGNS-IS (a Portuguese
tech startup). At the time, the company's flagship product was the
\gls{FFF}. The \gls{FFF} was a forest fire detection system, capable of
mostly autonomous and automatic operation.  The system was the first
application of \gls{DOAS} for fire detection, and for that it was
patented in 2007 (see~\cite{Vieira2007, Application2008}). The \gls{FFF}
is a remote sensing device that scans the horizon for the presence of a
smoke column, sequentially performing a chemical analysis of each
azimuth, using the Sun as a light source for its spectroscopic
operations~\cite{ValentedeAlmeida2017}.

The \gls{FFF} was deployed in several "habitats", both nationally
(Parque Nacional da Peneda-Gerês and Ourém) and internationally (Spain
and Brazil). One of the company's clients at the time was interested in
a pollution monitoring solution, and asked if the spectroscopic system
would be capable of performing such a task. The challenge resonated
through the company's structure and the idea that created this thesis
was born. The team then started reading about the concept of \gls{AP}
and how both populations and entities were concerned about it. It became
clear that, while there were already several methods to measure
\gls{AP}, there was a clear market drive for the development of a system
that could leverage the large area capabilities of a \gls{DOAS} device
while being able to provide a more spatially resolved "picture" of the
atmospheric status. With this in mind, the company managed to have the
investigation financed through a \gls{PT2020} funding opportunity. This
achievement was a clear validation of the project's goals and of the
need there was for a system with the proposed capabilities. It was,
however, not enough. \gls{FFF} was a very good starting point, but there
was still a lot of continuous research work needed before any of the
goals that had been set were achieved. This led to the publication of
this PhD project, in a tripartite consortium between FCT-NOVA, NGNS-IS
and the Portuguese Foundation for Science and Technology. Its main goal
was to develop an atmospheric monitoring system prototype that would be
able to spectroscopically map pollutant concentrations in a
two-dimensional way.

In April 2017, NGNS-IS was integrated in the Compta group, one of the
oldest IT groups operating in Portugal. Despite its age, this company is
one of the main presences in some of the most modern industrial fields,
like \gls{IOT} applications. \gls{ATMOS}'s pollutant tracing
capabilities made it an almost perfect fit in one of \gls{IOT}'s most
resounding niches, the \emph{Smart Cities} trend. Unfortunately, the
transition between one company and the other, regardless of the
project's adequacy, was anything but smooth. Almost two years later, in
the beginning of 2019, engulfed in a sea of endless bureaucracy and ill
intent on behalf of the managing governmental authorities (who seemed
always more interested in seeing the project fail than anything else),
\gls{ATMOS} was terminated and financing was cut.

\subsection{Research Questions}%
\label{sub:research_questions}

In Section~\ref{sub:context}, I have introduced the reasons which led
NGNS-IS to pursue the development of an atmospheric monitoring system,
and that what set it apart from other systems was the ability to
spectroscopically map pollutants concentrations using tomographic
methods, thus defining a primary objective for this thesis.

Two secondary objectives were born from the necessary initial research,
which had a very heavy influence over the adopted methods:
\begin{itemize}
    \item To use a tomographic approach for the mapping procedure;
    \item To ensure the designed system would be small and highly
        mobile;
    \item To use a single light collection point, minimizing material
        costs.
\end{itemize}

Taking all the above into account, we arrive at the main Research
Question (\gls{RQ}), presented in Table~\ref{tab:RQ1}.
\begin{table}[htpb]
    \centering
    \caption{Main research question.}
    \label{tab:RQ1}
    \begin{tabularx}{0.8\textwidth}{cX}
        \toprule
        \textbf{RQ1}&\emph{ How to design a miniaturized tomographic
        atmosphere monitoring system based on DOAS? }\\
        \bottomrule
    \end{tabularx}
\end{table}

This is the main research question. It gave rise to four other more
detailed research questions. These secondary questions allow a better
delimitation of the work at hand and are important complements to RQ1.
This questions are presented in Table~\ref{tab:sec_RQ}.

\begin{table}[htpb]
    \centering
    \caption{Secondary research questions.}
    \label{tab:sec_RQ}
    \begin{tabularx}{0.8\textwidth}{cX}
        \toprule
        \textbf{RQ1.1}&\emph{ What would be the best strategy
        for the system to cover a small geographic region? }\\
        \midrule
        \textbf{RQ1.2}&\emph{ What would be the necessary
        components for such a system? }\\
        \midrule
        \textbf{RQ1.3}&\emph{ How will the system acquire the
        data? }\\
        \midrule
        \textbf{RQ1.4}&\emph{ What should the tomographic
        reconstruction look like and how to perform it? }\\
        \bottomrule
    \end{tabularx}
\end{table}

\section{Problem Introduction}%
\label{sec:problem_introduction}

\acrlong{AP} poses an important threat to the human way of life. The
\gls{WHO} have estimated that 1 out of each 9 deaths in 2012 were
\gls{AP}-related and of these, 3 million were directly attributable to
outdoor \gls{AP} worldwide, most of which in developing countries (87\%
vs 82\% population). Although the European picture is not so dire as
this, the topic does cause concern. In 2016, there were an estimated
400.000 deaths due to \gls{AP} in Europe, 391.000 of which in the EU-28
space~\cite{Guerreiro2019}. An increased number of premature deaths is
sufficiently bad for treating this issue seriously, but the problems
brought forth by \acrlong{AP} do not end here. Not only are people dying
more, disabilities (namely respiratory) are more frequent, and so are
hospital visits. These two factors represent a decrease in productivity
and an increase in medical costs, which accrue to the huge burden that
\gls{AP} already represents to any society. In Europe, health impacts of
diesel emissions were estimated to be in the region of 60 billion
euros~\cite{CEDelft2018} for the year of 2016.

These impressive numbers have perspired onto the public opinion, which
is (now more than ever) concerned with the whole problem of \gls{AP}. In
fact, the subject is the considered by the public the second most
important environmental threat (after Climate Change, which is a very
related topic), and citizens throughout Europe have been partaking in
initiatives which aim to aid and incentivize air quality monitoring, as
well as raising awareness to the necessity of paying attention to this
issue and for behavioral changes. As tackling \acrlong{AP} and its
causes grows ever more popular, so does the political weight associated
with the subject, which in turn results in an increased number of
measures destined improve air quality. However, effective actions
against \gls{AP} require the approach to be intelligent and
knowledgeable, for the more we know, the better we can handle it. It is
thus the role of technology and technologists, to develop new ways in
which to measure, map, track and trace \gls{AP}, leveraging the power of
human intellect to combat this impending threat that is upon us.

% \section{Literature review}%
% \label{sec:literature_review}

% \subsection{Tomography}%
% \label{sub:tomography}

% \subsubsection{Reconstruction algorithms}%
% \label{ssub:reconstruction_algorithms}

% \begin{itemize}
%     \item Analytical
%         \begin{itemize}
%             \item FBP;
%             \item Fan-beam FBP.
%         \end{itemize}
%     \item Iterative
%         \begin{itemize}
%             \item SART;
%             \item SIRT;
%             \item MLEM;
%         \end{itemize}
% \end{itemize}

% \subsubsection{Geometries}%
% \label{ssub:geometries}

% \subsection{DOAS}%
% \label{sub:doas}

% \subsubsection{Background}%
% \label{ssub:background}

% \subsubsection{DOAS-tomography}%
% \label{ssub:doas_tomography}

% \section{Terms and scope of topics}%
% \label{sec:terms_and_scope_of_topics}

% \section{Knowledge gap}%
% \label{sec:knowledge_gap}

% \subsection{Smart air pollution monitoring}%
% \label{sub:smart_air_pollution_monitoring}

% \subsection{2d or 3d concentration mapping technologies}%
% \label{sub:2d_or_3d_concentration_mapping_technologies}

% \subsection{Low mobility of existing systems}%
% \label{sub:low_mobility_of_existing_systems}

% \section{Relevance of topic}%
% \label{sec:relevance_of_topic}

% \subsection{Smart cities}%
% \label{sub:smart_cities}

% \subsection{Air pollution monitoring in the future}%
% \label{sub:air_pollution_monitoring_in_the_future}

% \section{Research Questions}%
% \label{sec:research_questions}

% \section{Hypothesis}%
% \label{sec:hypothesis}

% \section{Methods}%
% \label{sec:methods}

% \subsection{Simulator}%
% \label{sub:simulator}

% \subsection{UAV Instruments}%
% \label{sub:uav_instruments}

% \subsection{Telescope and adaptations}%
% \label{sub:telescope_and_adaptations}

% \subsubsection{Model selection}%
% \label{ssub:model_selection}

% \subsubsection{Spectrometer coupling}%
% \label{ssub:spectrometer_coupling}

% \subsection{Experiments}%
% \label{sub:experiments}

% \subsection{SITL for trajectory programming}%
% \label{sub:sitl_for_trajectory_programming}

% \section{Findings}%
% \label{sec:findings}

% \section{Layout}%
% \label{sec:layout}














