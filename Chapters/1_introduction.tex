%!TEX root = ../thesis_rui_almeida.tex
%%%%%%%%%%%%%%%%%%%%%%%%%%%%%%%%%%%%%%%%%%%%%%%%%%%%%%%%%%%%%%%%%%%
%%%%%%%%%%%%%%%%%%%%%%%%%%%%%%%%%%%%%%%%%%%%%%%%%%%%%%%%%%%%%%%%%%%
%% chapter1.tex
%% Rui V. Almeida's thesis file
%%
%% Chapter with introduciton
%%%%%%%%%%%%%%%%%%%%%%%%%%%%%%%%%%%%%%%%%%%%%%%%%%%%%%%%%%%%%%%%%%%
\chapter{Introduction}%
\label{cha:introduction}

\section{Background, Motivation and Starting Points}%
\label{sec:background_and_motivation}

\subsection{Introduction}%
\label{sub:introduction}

This thesis describes the work that I have done in the past 4 years on
the design and development of a miniaturized system for atmospheric
monitoring based on optical spectroscopy. The project itself was the
major part of the \gls{ATMOS}, an initiative that was
contemplated with European funding through a \gls{PT2020} initiative and
came as a response to the growing weight that \gls{AP} has in the whole
Western world.

The potential impact of \gls{AP} on human health is amply documented.
Numerous papers have, for decades, established many links between air
quality and several common ailments like respiratory syndromes and
cardiovascular diseases. Similar connections have also been found
regarding the probability of gestational malformations and several types
of cancer. On a different level, and of perhaps less immediate concern,
are the effects that have been observed on ecosystems. Many times these
effects are difficult to predict (and timely mitigate) and in some cases
have been known to interfere with people's livelihood. In time, and if
not addressed, these interferences will certainly hinder economies and
limit the quality of life of populations globally. The severity of this
problem makes it clear that we need to tackle it intelligently, and this
approach requires that we can measure, trace and track \gls{AP}
effectively, which beckons engineers and scientists to create more
technology for this specific purpose.

Answering this call, with this work I have tried to create a reply to
the question of whether it would be possible to develop a
two-dimensional pollutant mapping tool, small enough to be fitted onto a
\gls{UAV}, which came to be a tomographically enabled design. To this
end, I have developed a simulation platform that computationally proves
the method's feasibility and confirmed through experiments that the
hypothesis on which the solution is based, regarding the use of
sequentially measured scattered sunlight as analogous to an artificial
light source is valid.

\subsection{Context}%
\label{sub:context}

The idea behind this thesis was born in 2015, at NGNS-IS (a Portuguese
tech startup). At the time, the company's flagship product was the
\gls{FFF}. The \gls{FFF} was a forest fire detection system, capable of
mostly autonomous and automatic operation.  The system was the first
application of \gls{DOAS} for fire detection, and for that it was
patented in 2007 (see~\cite{Vieira2007, Application2008}). The \gls{FFF}
is a remote sensing device that scans the horizon for the presence of a
smoke column, sequentially performing a chemical analysis of each
azimuth, using the Sun as a light source for its spectroscopic
operations~\cite{ValentedeAlmeida2017}.

The \gls{FFF} was deployed in several "habitats", both nationally
(Parque Nacional da Peneda-Gerês and Ourém) and internationally (Spain
and Brazil). One of the company's clients at the time was interested in
a pollution monitoring solution, and asked if the spectroscopic system
would be capable of performing such a task. The challenge resonated
through the company's structure and the idea that created this thesis
was born. The team then started reading about the concept of \gls{AP}
and how both populations and entities were concerned about it. It became
clear that, while there were already several methods to measure
\gls{AP}, there was a clear market drive for the development of a system
that could leverage the large area capabilities of a \gls{DOAS} device
while being able to provide a more spatially resolved "picture" of the
atmospheric status. With this in mind, the company managed to have the
investigation financed through a \gls{PT2020} funding opportunity. This
achievement was a clear validation of the project's goals and of the
need there was for a system with the proposed capabilities. It was,
however, not enough. \gls{FFF} was a very good starting point, but there
was still a lot of continuous research work needed before any of the
goals that had been set were achieved. This led to the publication of
this PhD project, in a tripartite consortium between FCT-NOVA, NGNS-IS
and the Portuguese Foundation for Science and Technology. Its main goal
was to develop an atmospheric monitoring system prototype that would be
able to spectroscopically map pollutant concentrations in a
two-dimensional way.

In April 2017, NGNS-IS was integrated in the Compta group, one of the
oldest IT groups operating in Portugal. Despite its age, this company is
one of the main presences in some of the most modern industrial fields,
like \gls{IOT} applications. \gls{ATMOS}'s pollutant tracing
capabilities made it an almost perfect fit in one of \gls{IOT}'s most
resounding niches, the \emph{Smart Cities} trend. Unfortunately, the
transition between one company and the other, regardless of the
project's adequacy, was anything but smooth. Almost two years later, in
the beginning of 2019, engulfed in a sea of endless bureaucracy and ill
intent on behalf of the managing governmental authorities (who seemed
always more interested in seeing the project fail than anything else),
\gls{ATMOS} was terminated and financing was cut.



\section{Problem Introduction}%
\label{sec:problem_introduction}

\acrlong{AP} poses an important threat to the human way of life. The
\gls{WHO} have estimated that 1 out of each 9 deaths in 2012 were
\gls{AP}-related and of these, 3 million were directly attributable to
outdoor \gls{AP} worldwide, most of which in developing countries (87\%
vs 82\% population). Although the European picture is not so dire as
this, the topic does cause concern. In 2016, there were an estimated
400.000 deaths due to \gls{AP} in Europe, 391.000 of which in the EU-28
space~\cite{Guerreiro2019}. An increased number of premature deaths is
sufficiently bad for treating this issue seriously, but the problems
brought forth by \acrlong{AP} do not end here. Not only are people dying
more, disabilities (namely respiratory) are more frequent, and so are
hospital visits. These two factors represent a decrease in productivity
and an increase in medical costs, which accrue to the huge burden that
\gls{AP} already represents to any society. In Europe, health impacts of
diesel emissions were estimated to be in the region of 60 billion
euros~\cite{CEDelft2018} for the year of 2016.

These impressive numbers have perspired onto the public opinion, which
is (now more than ever) concerned with the whole problem of \gls{AP}. In
fact, the subject is the considered by the public the second most
important environmental threat (after Climate Change, which is a very
related topic), and citizens throughout Europe have been partaking in
initiatives which aim to aid and incentivize air quality monitoring, as
well as raising awareness to the necessity of paying attention to this
issue and for behavioral changes. As tackling \acrlong{AP} and its
causes grows ever more popular, so does the political weight associated
with the subject, which in turn results in an increased number of
measures destined improve air quality. However, effective actions
against \gls{AP} require the approach to be intelligent and
knowledgeable, for the more we know, the better we can handle it. It is
thus the role of technology and technologists, to develop new ways in
which to measure, map, track and trace \gls{AP}, leveraging the power of
human intellect and ingenuity to combat this impending threat that is
upon us.

\section{Literature review}%
\label{sec:literature_review}

\subsection{Tomography}%
\label{sub:tomography}

Tomography is the cross-sectional imaging of an object through the use
of transmitted or reflected waves, captured by the object exposure to
the waves from a set of known angles. It has many different applications
in science, industry, and most prominently, medicine\todo{citatoins:
examples of industrial applications of tomography}. Since the invention
of the \gls{CT} machine in 1972, by Hounsfield~\cite{Gunderman2006},
tomographic imaging techniques have had a revolutionary impact, allowing
doctors to see inside their patients, without having to subject them to
more invasive procedures~\cite{Kak2001}.

Mathematical basis for tomography were set by Johannes Radon in 1917. At
the time, he postulated that  it is possible to represent a function
written in $\mathbb{R}$ in the space of straight lines, $\mathbb{L}$
through the function's line integrals. A line integral is an integral in
which the function that is being integrated is evaluated along a curved
path, a line. In the tomographic case, these line integrals represent a
measurement on a ray that traverses the \gls{ROI}.  Each set of line
integrals, characterized by an incidence angle, is called a projection
(see Figure~\ref{fig:projection}). To perform a tomographic
reconstruction, the machine must take many projections around the
object. To the set of projections arranged in matrix form by detector
and projection angle, we call sinogram. All reconstruction methods,
analytical and iterative, revolve around going from reality to sinogram
to image~\cite{Bruyant2002, Kak2001, Herman1973, Herman1995, Herman2009,
Defrise2003}.

\begin{figure}[htpb]
    \centering
    \includegraphics[width=0.5\textwidth]{img/png/projections.png}
    \caption{A schematic representation of a projection acquisition. In
    this image, taken from ~\cite{Herman2009}, the clear line that comes
    down at a diagonal angle is a projection.}
    \label{fig:projection}
\end{figure}

There are two broad algorithm families when it comes to tomographic
reconstruction, regarding the physics of the problem. It can involve
either non-diffracting sources (light travels in straight lines), such
as the X-Rays in a conventional \gls{CT} exam; or diffracting sources,
such as micro-waves or ultrasound in more research-oriented
applications~\cite{Kak2001}. In this document, I will not address the
latter family, since I will not be applying them in my work.

In any tomographic procedure, the first step is to gather information
from the target object. The first concept one requires for this is to
determine the problem's geometry. There are many different possible
geometry, however, there are two that are more important for this
thesis: parallel and fan-beam geometries, which are presented in
Figure~\ref{fig:geometries}. In the parallel case, there are as many
light sources as there are detectors. Light travels between the source
and the detector in straight lines, and the whole set rotates around the
object's location. Fan-beam geometries are characterized by having only
one light source which rotates around the target object. In this
geometry, a set of detectors are placed on the other side of the object,
and the lines (rays) between the source and the detectors describe a
fan, thus the name of the technique~\cite{Herman2009, Kak2001}.

As far as algorithms are concerned, there are two main types: analytical
and iterative. The first family includes the most famous algorithm for
these applications, the \gls{FBP}\todo{Reference to the chapter in which
this technique is explained}. Iterative algorithms work by iteratively
searching for a solution to the reconstruction equation, which is
basically an underdetermined system of equations~\todo{check this!}.
There are numerous algorithms that work in this way, but in my work, I
have identified three that are extensively used, both in the field of
atmospheric tomography and in medicine: \gls{ART}, \gls{SART}, and
\gls{MLEM}. The simulator that was developed as part of this project
uses the last two and \gls{FBP}.

\subsection{\gls{DOAS}}%
\label{sub:doas}

Since the beginning of the 20\textsuperscript{th} century, scientists
have been using spectroscopy to measure reactive trace gases in the
atmosphere, especially ozone. The basis for these applications were set
by Bouguer, Lambert and Beer, which have separately presented the law
(Lambert-Beer's) that determines the relationship between light
extinction and the concentration of an absorber, when it must traverse a
medium in which this absorber is present. \gls{DOAS} is one of the
methods that is applied for this purpose. It was developed in 1976, by
Perner and his colleagues~\cite{Perner1976}, to detect and quantify the
hydroxyl radical in the atmosphere. The book by Jochen Stütz and Ulrich
Platt~\cite{Platt2007} is considered by most researchers one of the most
important references in the field and is present in most bibliographies
of the literature in this subject (it is also one of the main references
in this thesis). Platt, in particular, has been working with the
technique since its beginning, as one of the elements of Perner's team
that published the article about the hydroxyl radical mentioned some
lines above~\cite{Perner1976}.

Besides \gls{DOAS}, Lambert-Beer's law is the basis of many quantitative
spectroscopy applications. However, most of these techniques are used in
a laboratory context, in which conditions are controlled and very well
known. Atmospheric studies do not have this luxury. In the open
atmosphere, there are a number of factors, like Mie and Rayleigh
scattering, atmospheric turbulence or thermal fluctuations in the
optical path that make outdoor spectral measurements more complicated.
\gls{DOAS} is able to circumvent these difficulties by measuring
differential absorptions, which is to say the difference in absorption
between two different wavelengths~\cite{Platt2007, Merlaud2013}.

There are two modalities for \gls{DOAS} experiments, Active and Passive,
which differ mainly on the use of artificial or natural light sources,
respectively. Both methods have their advantages and disadvantages.
Active systems are more similar to a bench spectroscopy experiment.
Conditions are more controlled (starting with the light source) and
therefore, results are usually more reliable and precise, not to mention
simpler to reach, since there is no need to account for complex physical
phenomena, like radiative transfer~\cite{Platt2007}. However, these
systems do require additional material, and many times entire
infrastructures have to be built around them~\cite{Pundt2005}. Passive
systems, on the other hand, can be comprised of just a computer, a
spectrometer and a telescope, making them instrumentally much simpler
than their active counterparts. This flexibility also comes with the
possibility to develop new interesting sub-techniques, like
\gls{maxdoas}, which allows (for instance) the determination of the
stratospheric contribution of a certain trace gas (in opposition to its
tropospheric contribution). The mathematical processing of the acquired
spectra, nonetheless, is much more complex.

\subsection{DOAS Tomography}%
\label{sub:doas_tomography}

DOAS tomography is a relatively new subject within the realm of
\gls{DOAS}. It consists in the application of tomographic methods to
reconstruct a two-dimensional or three-dimensional \emph{map} of the
concentrations of trace gases in study. The seminal paper that
originated this and other remote sensing tomographic techniques was
published by Shepp in xxxx~\todo{This is written in some of the
tomographic DOAS papers.}. It is not by any means a very populated
literary space. In a systematic review that was performed as part of a
course I took during the development of this PhD thesis~\todo{Reference
the section in which this paper is more thoroughly discussed}, I managed
to identify a total of 13 papers that were clearly about DOAS
tomography. In doing this review, I have alaso found that the largest
DOAS tomography study was performed in Germany, during the first years
of the 21\textsuperscript{st} century. This research campagin, called
BAB-II~\cite{Laepple2004}, aimed to measure and map traffic-related
concentrations for \gls{no2} in the motorway that goes between
Heidelberg and Mannheim. A more recent effort that is mention worthy is
the paper by the Stutz~\cite{Stutz2016}, in which the group created a
tomographic system that was able to perform as a fence  line monitor for
a refinery in Houston, Texas. In between the two studies, and right
after BAB-II, Erna Frins published her paper~\cite{Frins2006}, in which
she described the use of a \gls{maxdoas} system alternately pointed
towards sun-illuminated and dark targets in a tomographic manner.

Besides the birds eye view of the literary panorama of the DOAS
tomography field, this systematic review allowed me to understand two
important gaps in the technology being employed for this research. The
first is that all of the studies that were featured in my review used a
very low number of tomographic projections (some dozens of line
integrals). This is a problem because it is the single most important
factor for the resolution of any tomographic procedure, and although
resolution is not an absolutely critical factor in atmospheric analysis
(because the sizes of the target objects - gas plumes - are very large
and diffuse), it is an important system feature that should be improved.
Moreover, the second important pattern that was clear from my research
was that all but one of the described systems were fixed, and the one
that was mobile was composed of a minimum of two spectral acquisition
devices and had one of the lowest numbers of projections in all papers.
This is a very important gap. DOAS tomography has the ability not only
to measure but also to map pollutant concentrations and can be an
invaluable technique in the fight to understand and track the movement
of pollutant plumes, but the fact that the available systems require
dedicated infrastructure to operate and have no mobility at all may be
the most important factor leading to the almost non-existing investment
DOAS tomography systems.

More than half of the world's population is expected to be living in
cities by the year 2050, and in the next 10 years we will see an
increase in the number of so-called megacities of more than
30\%~\cite{CABI2019}. This puts a lot of pressure on governmental
agencies and municipalities (specially in the West) to "smarten up"
their urban infrastructures, so that cities can harbor their inhabitants
with reasonable quality of life for everyone. In fact, a recent report
by xxx concluded that in the next yyy years, states and cities will
spend more than 150 billion Euros on this kind of
infrastructure~\todo{Check these numbers}. The flexibility and mobility
that the system I am developing brings to the table are two very heavy
points in its favor as a pollution mapping tool which is unobtainable
with the traditional methods, whether \emph{in-situ} or remote.


\section{Research Questions}%
\label{sec:research_questions}

In Section~\ref{sub:context}, I have introduced the reasons which led
NGNS-IS to pursue the development of an atmospheric monitoring system,
and that what set it apart from other systems was the ability to
spectroscopically map pollutants concentrations using tomographic
methods, thus defining a primary objective for this thesis.

Two secondary objectives were born from the necessary initial research,
which had a very heavy influence over the adopted methods:
\begin{itemize}
    \item To use a tomographic approach for the mapping procedure;
    \item To ensure the designed system would be small and highly
        mobile;
    \item To use a single light collection point, minimizing material
        costs.
\end{itemize}

Taking all the above into account, we arrive at the main Research
Question (\gls{RQ}), presented in Table~\ref{tab:RQ1}.
\begin{table}[htpb]
    \centering
    \caption{Main research question.}
    \label{tab:RQ1}
    \begin{tabularx}{0.8\textwidth}{cX}
        \toprule
        \textbf{RQ1}&\emph{ How to design a miniaturized tomographic
        atmosphere monitoring system based on DOAS? }\\
        \bottomrule
    \end{tabularx}
\end{table}

This is the main research question. It gave rise to four other more
detailed research questions. These secondary questions allow a better
delimitation of the work at hand and are important complements to RQ1.
This questions are presented in Table~\ref{tab:sec_RQ}.

\begin{table}[htpb]
    \centering
    \caption{Secondary research questions.}
    \label{tab:sec_RQ}
    \begin{tabularx}{0.8\textwidth}{cX}
        \toprule
        \textbf{RQ1.1}&\emph{ What would be the best strategy
        for the system to cover a small geographic region? }\\
        \midrule
        \textbf{RQ1.2}&\emph{ What would be the necessary
        components for such a system? }\\
        \midrule
        \textbf{RQ1.3}&\emph{ How will the system acquire the
        data? }\\
        \midrule
        \textbf{RQ1.4}&\emph{ What should the tomographic
        reconstruction look like and how to perform it? }\\
        \bottomrule
    \end{tabularx}
\end{table}

\section{Hypothesis}%
\label{sec:hypothesis}

As stated in Section~\ref{sec:research_questions}, the main goal of my
work was to provide an answer to the question of how to design a
miniaturized tomographic atmosphere monitoring system, based on
\gls{DOAS}.  This sentence is the most important point of the project.
Every development from this point forward stems from it and is motivated
by it. It also requires some deconstruction in order to understand the
true scope of the matter. First, it is a miniaturized device. This means
that besides all the habitual requirements (performance, function,
adequacy, compatibility with the other components, safety and security)
in defining components in an engineering project, one must also keep in
mind the footprint of each component, and how much it weighs. Second, it
is a tomographic system, which means that not only the device must be
able to take line integrals from some kind of medium, it must also be
able to describe a predefined trajectory, with admissible levels of
geometric error, which complies to a certain projection geometry.
Otherwise, one would not be able to apply a tomographic reconstruction
routine to obtain the map of the target species concentrations. Finally
the system is supposed to monitor the atmosphere in some way. Now, as
implied in the same sentence, \gls{DOAS} is the technique that I am
trying to apply in this system. But as I point out in
Section~\ref{sub:doas} and with more depth in
Section~\ref{sec:lit_review_doas} , there are two families of
\gls{DOAS}, and within them, many sub-techniques. This system which I am
developing is based on the hypothesis that we can use an almost hybrid
approach to \gls{DOAS}: passive \gls{DOAS} instrumental simplicity and
active \gls{DOAS} retrieval simplicity. As illustrated in
Figure~\ref{fig:hypothesis}, one could use a scattered sunlight
measurement as a light source for a \gls{DOAS} analysis, provided
distances between the two points are kept small, optical densities are
low (clear atmosphere), and both spectral measurements are taken in the
same angle. With this process, we would only effectively be using as
projections the spectral measurements of the \gls{ROI}. This hypothesis,
while not being mentioned in the library directly, has already been
hinted at in several papers, namely references~\cite{Frins2006,
Casaballe2017, Johansson2009}.

\begin{figure}[htpb]
    \centering
    \missingfigure{}
    \caption{Hypothesis schematic. Light captured at point A is used as
        a light source (or $I_0$) for the light captured at point B,
        just as if using an artificial light source. As long as
        distances are kept small and the optical thickness is low,
        scattering will be negligible.  This is a huge simplification
        over the traditional passive \gls{DOAS} analysis process, as
        explained in Section~\ref{sec:lit_review_doas}.}
    \label{fig:hypothesis}
\end{figure}

\section{Methods and Findings}%
\label{sec:methods}

\todo[inline]{Consider rewriting this section.}

The system proposed in this thesis is a conjunction of sub-components
that work together towards the final goal. The first division is between
what is a physical, tangible component and what is just software
related. On the first level, developing this project involved selecting
the components for a custom made \gls{UAV} to which the spectroscopic
system will be attached to. In addition to this, I also had to select
the on-board electronics module, which is comprised of the flight
controller, which takes care of the device's movement; and the on-board
computer, which handles the acquisition of data via the optical module,
an assembly of spectrometers and telescopes connected through a
custom-built coupling part. All this assembly will perform according to
what is programmed using ArduCopter's SITL programming
suite~\todo{citation}.

On a software level, developing this project implied the creation of a
tomographic simulation tool, written in Python (and more particularly,
NumPy)~\cite{Python, Oliphant2007}. This tool was coded considering the
hypothesis described in Section~\ref{sec:hypothesis}, and includes the
description of the trajectory that the system must conduct in order to
reconstruct the target species concentration maps. Regarding the
reconstruction itself, this tool includes custom-built routines for the
\gls{FBP} algorithm and the \gls{MLEM} algorithm, and applies a SciPy
library~\todo{citation} method to perform \gls{SART} reconstruction.
Moreover, the simulation platform also takes into consideration
geometric and reconstruction errors, proper to this kind of system.

The rest of this thesis describes my best efforts in developing a
spectral system that would provide an affirmative reply to the research
question, taking into account the literary gaps that we have found in
our literature review. In doing this, I have successfully built a
software simulation tool that computationally proves that our
tomographic assumptions are reasonable and allow the concentration
mapping of any trace species that can be targeted by the \gls{DOAS}
technique. On an experimental level, I have also managed to validate the
hypothesis described in Section~\ref{sec:hypothesis}.

\section{Layout}%
\label{sec:layout}

After this introduction, in Chapter~\ref{cha:literature_review}, we can
find a more thorough literature review than the one presented in
Section~\ref{sec:literature_review}. After that, I have included the
chapter in which I present and discuss the methods employed in this
project's development (Chapter~\ref{cha:methods}). Finally, in
Chapter~\ref{cha:conclusions}, I present what could be concluded as a
result of this project.
















