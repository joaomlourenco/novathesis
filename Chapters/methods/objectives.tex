%!TEX root=../../thesis_rui_almeida.tex
\section{Section Objectives}%
\label{sec:section_objectives}

\begin{itemize}
    \item Summarise, explain and recount how the question was answered;
    \item Discuss the previous point in details;
    \item Why are techniques relevant;
    \item How were the techniques used;
\end{itemize}

\section{Questions the readers should be able to answer once they've
read the section}%
\label{sec:questions_the_readers_should_be_able_to_answer_once_they_ve_read_the_section}
\begin{itemize}
    \item What are the results;
    \item How do the findings relate to previous studies;
    \item Was there anything surprising that did not go as planned;
    \item Why the presented conclusion have been reached;
    \item Explain results;
\end{itemize}


\section{How was the question answered?}%
\label{sec:how_was_the_question_answered_}
\textbf{Question}

How to design a miniaturised tomographic atmosphere monitoring system
based on DOAS?

\textbf{How was the question addressed?}

\begin{description}
    \item[Projection-based hypothesis:]first step was to design an
        information gathering approach. Ours was based on the definition
        of a particular trajectory for the measurement device. This had
        to be validated in computational and mathematical sense;
    \item[Measurement based hypothesis:] although it is clearly implied
        by Lambertian theory, there is no literature that I know of that
        points to the fact that my measurements are correct. Therefore,
        the assumption that we can measure columns densities between two
        points sequentially must be tested;
\end{description}

\section{Detailed presentation of projection-based hypothesis}%
\label{sec:detailed_presentation_of_projection_based_hypothesis}

\begin{description}
    \item[Hypothesis:]
        \begin{itemize}
            \item A tomographic atmospheric measurement entails
                capturing projections in many different angles;
            \item The typical atmospheric tomographic DOAS application
                uses serveral tens of projections for reconstruction;
            \item A drone moving in a circular trajectory would be able
                to capture an array of fan-beam projections with a
                specially motorised spectroscopic assembly;
            \item In theory, this would be sufficient for
                reconstruction;
            \item For simplicity, in this application we use the
                re-sorting reconstruction algorithm, which applies the
                FBP algorithm to a synthetically created parallel beam
                sinogram;
        \end{itemize}
    \item[How is the trajectory defined:\\]
        \begin{itemize}
            \item circular;
            \item one stop at each $\alpha$ degrees. At each one:
                \begin{itemize}
                    \item gymbal points to a number of directions
                        (parameter) at regular intervals (parameter),
                        taking one measurement at a time;
                \end{itemize}
            \item on a second moment, the drone moves to the entry
                points of the rays that have been capturesd by the
                equipment, taking a measurement at each one, in the same
                direction;
            \item use the measurement hypothesis to determine the
                column densities between 1\textsuperscript{st}
                measurement moment and the 2\textsuperscript{nd}.
        \end{itemize}
\end{description}

\section{Calculation of Projections}%
\label{sec:calculation_of_projections}

\begin{enumerate}
    \item Discretise the Region Of Interest
\end{enumerate}
