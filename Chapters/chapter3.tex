%!TEX root = ../template.tex
%%%%%%%%%%%%%%%%%%%%%%%%%%%%%%%%%%%%%%%%%%%%%%%%%%%%%%%%%%%%%%%%%%%%
%% chapter3.tex
%% NOVA thesis document file
%%
%% This chapter includes a Literary Review.
%%%%%%%%%%%%%%%%%%%%%%%%%%%%%%%%%%%%%%%%%%%%%%%%%%%%%%%%%%%%%%%%%%%%
\chapter{Literature Review}
\label{cha:literature_review}

In this chapter, I provide a literary review on the three most important
subjects for the work of this thesis: \gls{AP}, tomographic algorithms
and \gls{DOAS} tomography instrumentation.

\section{Air pollution and pollutants}%
\label{sec:air_pollution_and_pollutants}

As stated in Section~\ref{sec:research_question}, the definition of
\gls{AP} is dependent on the context. Here, I will focus especially on
the effects of pollutants on human health. Whether these effects are the
most significant problems stemming from \gls{AP} is debatable (climate
change is mostly caused by anthropogenic production of greenhouse gases,
which are pollutants) but for this system and its intended uses, health
effects are definitely more prominent. Human health implications of a
polluted atmosphere are documented in very numerous studies throughout
the literature. In this document, I will only present a small number of
representative reviews and reports.

In 2004, \gls{WHO} published a report summarizing what was then the most
recent information on health effects of air pollution over Europe. This
review concluded that, even with all the regulations on \gls{AP} put in
place by the European authorities, its levels were still posed a
considerable burden on health throughout
Europe~\cite{WorldHealthOrganisationEurope2004}.

Although there are several hundred potentially harmful components
already that have already been found in the atmosphere, this report
addresses only \gls{PM}, ground level Ozone and Nitrogen Dioxide. As
many other studies had found, this Systematic Literature Review
(\gls{SLR}) identified several short-term and long-term exposure effects
for the three pollutants. The study found that short-term exposure to
all three substances were responsible for an increase in mortality and
hospital admissions, and that both \gls{PM} and O$_3$ increased the
population's usage of medication. Long-term exposure to all three
components have adverse pulmonary effects, but \gls{PM} have many other
negative effects. The most important of them a reduction in life
expectancy, which the authors attribute to cardiopulmonary mortality and
lung cancer.

Particulate Matter are described as airborne solid particles or
droplets. These particles vary in size, origin and composition, however,
it is usual to classify them by size, since that is what governs
particle deposition in the respiratory system. Urban \gls{PM} are
usually divided into three categories: coarse, fine and ultrafine.
Convention dictates that coarse particles have an aerodynamic diameter
of less that 10$\mu$m (PM$_{10}$), fine particles less than 2.5$\mu$m
(PM$_{2.5}$). In this review, the authors stated that the role that
ultrafine particles play in human health is still undetermined, but they
noted that coarse and (especially) fine particles are highly correlated
with an increase in mortality and the prevalence of respiratory
syndromes. 






\section{Tomographic algorithms and reconstruction techniques}%
\label{sec:tomographic_algorithms_and_reconstruction_techniques}

\subsection{Introduction}%
\label{sub:introduction}

Tomography is the cross-sectional imaging of an object through the use
of transmitted or reflected waves, captured by the object exposure to
the waves from a set of known angles. It has many different applications
in science, industry, and most prominently, medicine. Since the
invention of the Computed Tomography (\gls{CT}) machine in 1972, by
Hounsfielf~\cite{Gunderman2006}, tomographic imaging techniques have had
a revolutionary impact, allowing doctors to see inside their patients,
without having to subject them to more invasive
procedures~\cite{Kak2001a}.

Mathematical basis for tomography were set by Johannes Radon in 1917. At
the time, he postulated that  it is possible to represent a function
written in $\mathbb{R}$ in the space of straight lines, $\mathbb{L}$
through the function's line integrals. A line integral is an integral in
which the function that is being integrated is evaluated along a curved
path, a line. In the tomographic case, these line integrals represent a
measurement on a ray that traverses the Region Of Interest (\gls{ROI}).
Each set of line integrals, characterized by an incidence angle, is
called a projection (see Figure~\ref{fig:projection}). To perform a
tomographic reconstruction, the machine must take many projections
around the object. To the set of projections arranged in matrix form by
detector and projection angle, we call sinogram. All reconstruction
methods, analytical and iterative, revolve around going from reality to
sinogram to image~\cite{Bruyant2002, Kak2001, Herman1973, Herman1995,
Herman2009, Defrise2003}.

\begin{figure}[htpb]
    \centering
    \missingfigure{Projection Schematic}
    \caption{A schematic representation of a projection.}
    \label{fig:projection}
\end{figure}

There are two broad algorithm families when it comes to tomographic
reconstruction, regarding the physics of the problem. The problem can
involve either non-diffracting sources (light travels in straight
lines), such as the X-Rays in a conventional \gls{CT} exam; or
diffracting sources, such as micro-waves or ultrasound in more
research-oriented applications. In this document, I will not address the
latter family, since I will not be applying them in my work. In the next
few paragraphs, I will discuss the first family of algorithms, and
describe how an image can be reconstructed from an object's projections
when the radiation source is non-diffracting.

Let's consider the case in which we deal with a single ray of solar
light entering the atmosphere at a given point. Since the atmosphere
contains numerous absorbents and comparable atmospheric effects, the ray
changes from the point where it enters the atmosphere to the point at
which it is measured by a detector. Total absorption will depend on the
pollutant species, their cross-section and their concentration, since it
obeys Lambert-Beer's law. Looking from another angle, this absorption
is also the line integral that we will use to reconstruct our image.
With \gls{DOAS}, it is possible to measure several pollutants at the
same time, but for simplicity (and since it is one of the most studied
compounds in the field), let's consider that the single pollutant in our
atmospheric mixture is NO$_2$.

\subsection{Initial Considerations}%
\label{sub:initial_considerations}


The problem of tomographic reconstruction can be approached in a number
of ways, depending mostly on the authors. In my literary search, I have
found that Kak and Slaney~\cite{Kak2001} have certainly explained this
problem in one of the clearer ways available. Therefore, I shall base
the rest of my presentation in their writings, and complement with other
authors' notes wherever necessary.

Considering the coordinate system displayed in
Figure~\ref{fig:coordinates}. In this schematic, the object is
represented by the function $f(x, y)$. The  $(\theta, t)$ parameters can
be used to define any line in this schematic. Line AB in particular can
be written:

\begin{equation}
    \label{eq:lineAB}
    x \cdot \cos(\theta) + y \cdot \sin(\theta) = t
\end{equation}

\begin{figure}[htpb]
    \centering
    \missingfigure{Coordinates system}
    \caption{Schematic representation for coordinate setting.}
    \label{fig:coordinates}
\end{figure}

And if we were to write a line integral along this line, it would look
like Equation~\ref{eq:lineABIntegral}, the Radon transform of function
$f(x, y)$:

\begin{equation}
    \label{eq:lineABIntegral}
    P_{\theta}(t) = \int_{-\infty}^{\infty} f(x, y) \cdot \delta(x \cdot
    \cos(\theta) + y \cdot \sin(\theta) - t) dxdy
\end{equation}

Where $\delta$, the delta function, is defined in
Equation~\ref{eq:delta}.

\begin{equation}
    \label{eq:delta}
    \delta (\phi) =  
    \begin{cases}
            1, & \phi = 0\\
            0, & otherwise
    \end{cases}
\end{equation}

As I have mentioned previously, a projection is a set of line integrals
such as $P_{\theta}(t)$. Geometry plays a very important role in how the
integrals are written and solved for reconstruction. The simplest case
is the one where the set is acquired in a row, describing what is called
a parallel geometry. Another more complex case is when a single point
source is used as origin for all rays, forming a fan. This is called a
fanbeam array. There are other possible geometries, but they fall out of
the scope of this work and will therefore not be addressed any further.

\subsection{The Fourier Slice Theorem}%
\label{sub:the_fourier_slice_theorem}

The Fourier Slice Theorem (\gls{FST}) is the most important component of
the most important algorithm in tomographic inversion, the Filtered
BackProjection algorithm (\gls{FBP}). \gls{FST} is based on the equality
relation between the 
two-dimensional Fourier Transform (\gls{FT}) of the object function and
the one-dimensional \gls{FT} of the object's projection at an angle
$\theta$. Let's start by writing the 2D \gls{FT} for the object
function, Equation~\ref{eq:objectFT}, and the 1D \gls{FT} of projection
P$_\theta$, in Equation~\ref{eq:1dFTproj}.

\begin{equation}
    \label{eq:objectFT}
    F(u, v) = \int_{-\infty}^{\infty} \int_{-\infty}^{\infty} f(x, y)
    \cdot \exp \left [ -j2\pi (ux + vy) \right ] dx dy 
\end{equation}

\begin{equation}
    \label{eq:1dFTproj}
    S_{\theta}(\omega) = \int_{-\infty}^{\infty} P_{\theta} \cdot \exp\left[
    -j2 \pi \omega t \right]
\end{equation}

For simplicity, let's consider the 2D \gls{FT} at the line defined by
$v=0$ in the frequency domain. We rewrite the 2D \gls{FT} integral as:

\begin{equation}
    \label{eq:v0}
    F(u, 0) = \int_{-\infty}^{\infty} \int_{-\infty}^{\infty} f(x, y)
    \cdot \exp \left[  -j 2\pi  \omega ux \right] dx dy
\end{equation}

Notice that $y$ is not present in the phase factor of the \gls{FT}
expression anymore, and this means we can rearrange the integral as:

\begin{equation}
    \label{eq:v02}
    F(u, 0) = \int_{-\infty}^{\infty} \left[ \mathbf{\int_{-\infty}^{\infty}
    f(x, y) dy }\right] \cdot \exp \left[  -j 2\pi  \omega ux \right] dx 
\end{equation}

Now, the \textbf{bold} part of Equation~\ref{eq:v02} is similar to
Equation~\ref{eq:lineABIntegral}. It is precisely that equation,
considering $\theta=0$ and a constant value of $x$, as in
Equation~\ref{eq:p0}.

\begin{equation}
    \label{eq:p0}
    P_{\theta=0} (x) = \int_{-\infty}^{\infty} f(x, y) dy
\end{equation}

This in turn can be substituted in Equation~\ref{eq:v02}, finally
arriving at:

\begin{equation}
    \label{eq:FTP}
    F(u, 0) = \int_{-\infty}^{\infty} P_{\theta=0} (x) \cdot \exp \left[
    -j 2\pi ux \right] dx
\end{equation}

And this is the one-dimensional \gls{FT} for the projection at angle
$\theta=0$. Finally, the enunciation of the Fourier Slice Theorem:
\begin{center}
    \textbf{
    \emph{
        The Fourier Transform of a parallel projection  of an image $f(x,
        y)$ taken at angle $\theta$ gives a slice of the two-dimensional
        Fourier Transform, $F(u, v)$, subtending an angle $\theta$ with the
        $u$-axis (see Figure~\ref{fig:fst})
    }}
\end{center}

\begin{figure}[htpb]
    \centering
    \missingfigure{Fourier Slice Theorem}
    \caption{The \gls{FST}, a schematic representation.}
    \label{fig:fst}
\end{figure}

\subsection{The Filtered BackProjection Algorithm}%
\label{sub:the_filtered_backprojection_algorithm}

If one takes the \gls{FST} into account, the idea behind the \gls{FBP}
seems to appear almost naturally. Say one has a single projection and
its Fourier transform. From the \gls{FST}, this projection is the same
as the object's two-dimensional \gls{FT} in a single line. A crude
reconstruction of the original object would result if someone were to
place this projection in its right place in the Fourier domain and then
perform a two-dimensional \gls{IFT}, while assuming every other
projection to be 0. The result, in the image space, would be as if
someone had smeared the object in the projections direction.

What is really needed for a correct reconstruction is to do this many
times, with many projections. This brings a problem with the method:
smearing the object in all directions will clearly produce a wrong
\emph{accumulation} in the center of the image, since every projection
passes through the middle (remember we are still talking about parallel
geometry projections) and are summed on top of each other, but on the
outer edges, this does not occur. If one does not address this, the
image intensity levels in the reconstructed image will be severely
overestimated in the center and underestimated in the edges (due to
normalization). The solution is conceptually easy: we multiply the
Fourier transform by a weighting filter proportional to its frequency
($\omega$) and that encompasses its relevance in the global scheme of
projections. If there are $K$ projections, then it is adequate for this
value to be $\frac{2\pi\lvert\omega\rvert}{K}$. As an algorithm,
\gls{FBP} can be written as in Algorithm~\ref{alg:fbp}.
\begin{algorithm}
    \caption{The Filtered BackProjection Algorithm}
    \label{alg:fbp}
    \begin{algorithmic}
        \FORALL{$\theta, \theta \in \left\{0..180,
        \frac{180}{K}\right\}$}
        \STATE{Measure projection $P_{\theta}(t)$;}
        \STATE{FT($P_{\theta}(t)$), rendering $S_{\theta}(\omega)$}
        \STATE{Multiply by $\frac{2\pi\lvert{\omega}\rvert}{K}$;}
        \STATE{Sum the \gls{IFT} of the result in the image space.}
    \ENDFOR
    \end{algorithmic}
\end{algorithm}






\section{DOAS tomography instrumentation}%
\label{sec:doas_tomography_instrumentation}

\subsection{Active DOAS tomography}%
\label{sub:active_doas_tomography}

\subsection{Passive DOAS tomography}%
\label{sub:passive_doas_tomography}










% This Chapter aims at exemplifying how to do common stuff with \LaTeX. We also show some stuff which is not that common! ;)

% Please, use these examples as a starting point, but you should always consider using the \emph{Big Oracle} (aka, \href{http://www.google.com}{Google}, your best friend) to search for additional information or alternative ways for achieving similar results.

% \section{Document Structure} % (fold)
% \label{sec:document_structure}

% % section document_structure (end)


% \section{Dealing with Bibliogrpahy} % (fold)
% \label{sec:dealing_with_bibliogrpahy}

% % section dealing_with_bibliogrpahy (end)


% \section{Inserting Tables} % (fold)
% \label{sec:inserting_tables}

% % section inserting_tables (end)


% \section{Importing Images} % (fold)
% \label{sec:importing_images}

% % section importing_images (end)


% \section{Floats, Figures and Captions} % (fold)
% \label{sec:floats_figures_and_captions}

% % \subsection{Inserting Figures Wrapped with text} % (fold)
% % \label{ssec:inserting_images_wrapped_with_text}
% % 
% % You should only use this feature is \emph{really} necessary. This means, you have a very small image, that will look lonely just with text above and below.
% % 
% % In this case, you must use the \verb!wrapfiure! package.  To use \verb!wrapfig!, you must first add this to the preamble:
% % 
% % \begin{wrapfigure}{l}{2.5cm}
% %   \centering
% %     \includegraphics[width=2cm]{snowman-vectorial}
% %   \caption{A snow-man}
% % \end{wrapfigure}	
% % 
% % \noindent\verb!\usepackage{wrapfig}!\\
% % This then gives you access to:\\
% % \verb!\begin{wrapfigure}[lineheight]{alignment}{width}!\\
% % Alignment can normally be either ``l'' for left, or ``r'' for right. Lowercase ``l'' or ``r'' forces the figure to start precisely where specified (and may cause it to run over page breaks), while capital ``L'' or ``R'' allows the figure to float. If you defined your document as twosided, the alignment can also be ``i'' for inside or ``o'' for outside, as well as ``I'' or ``O''. The width is obviously the width of the figure. The example above was introduced with:
% % \lstset{language=TeX, morekeywords={\begin,\includegraphics,\caption}, caption=Wrapfig Example, label=lst:latex_example}
% % \begin{lstlisting}
% % 	\begin{wrapfigure}{l}{2.5cm}
% % 	  \centering
% % 	    \includegraphics[width=2cm]{snowman-vectorial}
% % 	  \caption{A snow-man}
% % 	\end{wrapfigure}	
% % \end{lstlisting}

% % subsection inserting_images_wrapped_with_text (end)

% % section floats_figures_and_captions (end)

% \lipsum[1-3]

% \begin{figure}[htbp]
%   \centering
%   \subcaptionbox{One sub-figure\label{fig:leftsubfig}}%
%     {\includegraphics[width=0.5\linewidth]{knitting-vectorial}}%
%   \subcaptionbox{Another sub-figure\label{fig:rightsubfig}}%
%     {\includegraphics[width=0.5\linewidth]{knitting-vectorial}}%
%   \caption{A figure with two sub-figures!}
%   \label{fig:fig2subfig}
% \end{figure}

% \textbf{And this is a small text that references the Figure~\ref{fig:fig2subfig} and its Subfigures~\ref{fig:leftsubfig} and~\ref{fig:rightsubfig}.}

% \lipsum[1-3]


% \section{Text Formatting} % (fold)
% \label{sec:text_formatting}

% % section text_formatting (end)


% \section{Generating PDFs from \LaTeX} % (fold)
% \label{sec:generating_pdfs_from_latex}

% \subsection{Generating PDFs with pdflatex} % (fold)
% \label{ssec:generating_pdfs_with_pdflatex}

% You may create PDF files either by using \verb!latex! to generate a DVI file, and then use one of the many DVI-2-PDF converters, such as \verb!dvipdfm!.

% Alternatively, you may use \verb!pdflatex!, which will immediately generate a PDF with no intermediate DVI or PS files. In some systems, such as Apple, PDF is already the default format for \LaTeX. I strongly recommend you to use this approach, unless you have a very good argument to go for \verb!latex! + \verb!dvipdfm!.

% A typical pass for a document with figures, cross-references and a bibliography would be:
% \begin{verbatim}
% $ pdflatex template
% $ bibtex template
% $ pdflatex template
% $ pdflatex template
% \end{verbatim}
% You will notice that there is a new PDF file in the working directory called \verb!template.pdf!. Simple :)

% Please note that, to be sure all table of contents, cross-references and bibliographic citations are up-to-date, you must run \verb!latex! once, then \verb!bibtex!, and then \verb!latex! twice.
% % section generating_pdfs_with_pdflatex (end)

% \subsection{Dealing with Images} % (fold)
% \label{sub:dealing_with_images}

% You may process the same source files with both \verb!latex! or \verb!pdflatex!. But, if your text include images, you must be careful. \verb!latex! and \verb!pdflatex! accept images in different (exclusive) formats.  For \verb!latex! you may use EPS ou PS figures. For \verb!pdflatex! you may use JPG, PNG or PDF figures.  I strongly recommend you to use PDF figures in vectorial format (do not use bitmap images unless you have no other choice).
% % subsection dealing_with_images (end)


% \subsection{Creating Source Files Compatible with both latex and pdflatex} % (fold)
% \label{ssec:creating_source_files_compatible_with_both_latex_and_pdflatex}

% Do not include the extension of the file in the \verb!\includegraphics! command. E.g., use\\
% \verb!\includegraphics{sonwman}!\\
% and not\\
% \verb!\includegraphics{sonwman.eps}!.\\
% If you use the first form, \verb!latex! or \verb!pdflatex! will add an appropriate file extension.

% This means that, if you plan to use only \verb!pdflatex!, you need only to keep (preferably) a PDF version of all the images. If you plan to use also \verb!latex!, then you also need an EPS version of each image.
% % subsection creating_source_files_compatible_with_both_latex_and_pdflatex (end)

% % section generating_pdfs_from_latex (end)


% \newpage

% {\Large To be included in the sections above}\\

% Para fazer citações, deverá usar-se a chave da referência no ficheiro BibTeX. Se for uma única referência~\cite{Artho04}, usar um ``\verb!~!'' para ligar o \verb!\cite{...}! à palavra que o precede (\ldots\verb!referência~\cite{Artho04}!).  Caso queira fazer múltiplas citações~\cite{Shavit95,Silberschatz06,Moss85}, deverá agrupá-las dentro de um úinico \verb!\cite{...}!.

% Note que o ficheiro de bibliografia pode ter tantas entradas quantas quiser. Apenas aquelas cuja chave seja referenciada no texto é que serão incluidas na listagem de bibliografia.


% Footnotes\footnote{This is a simple footnote.} will be numbered and shown in the bottom of the page.


% A Tabela~\ref{tab:hla:results} ilustra alguns conceitos importantes associados à contrução de tabelas:
% \begin{asparaenum}[i)]
% 	\item Não usar linhas verticais;
% 	\item A legenda deve ficar por cima da tabela;
% 	\item Usar as macros \verb!\toprule!, \verb!\midrule! e \verb!\bottomrule! para fazer a linha horizontal superior, interiores e inferior, respectivamente.
% \end{asparaenum}
 
% \begin{table}[ht]
% 	\caption{Test results summary.}
% 	\label{tab:hla:results}
% \centering
% \begin{tabular}{lccccc}
% 	\toprule
% 	\multicolumn{1}{c}{\textbf{Test}} 	& \textbf{Anomalies}	& \textbf{Warnings}	& \textbf{Correct} 	& \textbf{Categories}		& \textbf{Missed} \\
% 	\midrule
% \cite{Beckman08}~Connection 	& 2 & 2	& 1	& \emph{C}				& 1 \\
% \cite{Artho03}~Coordinates'03 	& 1	& 4	& 1	& \emph{2B, 1C}			& 0 \\
% \cite{Artho03}~Local Variable	& 1	& 2	& 1	& \emph{A}				& 0 \\
% \cite{Artho03}~NASA				& 1	& 1	& 1	& ---					& 0 \\
% \cite{Artho04}~Coordinates'04	& 1	& 4	& 1	& \emph{3C}				& 0 \\
% \cite{Artho04}~Buffer			& 0	& 7	& 0	& \emph{2A, 1B, 2C, 2D}	& 0 \\
% \cite{Artho04}~Double-Check		& 0	& 2	& 0	& \emph{1A, 1B}			& 0 \\
% \cite{Flanagan04}~StringBuffer	& 1	& 0	& 0	& ---					& 1 \\
% \cite{Praun03}~Account			& 1	& 1	& 1	& ---					& 0 \\
% \cite{Praun03}~Jigsaw			& 1	& 2	& 1	& \emph{C}				& 0 \\
% \cite{Praun03}~Over-reporting	& 0	& 2	& 0	& \emph{1A, 1C}			& 0 \\
% \cite{Praun03}~Under-reporting	& 1	& 1	& 1	& ---					& 0 \\
% \cite{IBM-Rep}~Allocate Vector	& 1	& 2	& 1	& \emph{C}				& 0 \\
% Knight Moves					& 1	& 3	& 1	& \emph{2B}				& 0 \\
% 	\midrule
% 	\textbf{Total}			& \textbf{12}		& \textbf{33}		& \textbf{10}			& \textbf{5A, 6B, 10C, 2D}	& \textbf{2} \\
% 	\bottomrule
% \end{tabular}
% \end{table}


% As figuras a inserir no documento deverão ser de qualidade, preferencialmente em formato vectorial (PDF vectorial) e não em \emph{bitmap} (PNG, JPG, etc). As imagens \emph{bitmap} (Figura~\ref{fig:Figuras_Tree_silhouettes-bitmap}) não escalam bem e têm reflexos negativos na qualidade do seu docuemnto.  Pelo contrário, as imagens \emph{vectoriais} {Figura~\ref{fig:Figuras_Tree_silhouettes-vectorial}} escalam muito tanto quanto o necessário sem degradar a qualidade da imagem.

% Só deve usar \emph{screenshots} se não tive mesmo nenhuma alternativa.  Em vez e gerar um \emph{screenshot}, tente usar uma impressora virtual PDF e imprimir para um ficheiro PDF. Regra geral obterá um PDF vetorial. Mesmo que o seu PDF contenha imagens, elas terão sempre qualidade maior ou igual à que obteria com um \emph{screenshot}.


% \begin{figure}[htbp]
% 	\centering
% 	\includegraphics[height=1in]{snowman-bitmap}
% 	\includegraphics[height=3in]{snowman-bitmap}
% 	\includegraphics[height=6in]{snowman-bitmap}
% 	\caption{Imagem em formato \emph{bitmap} (JPG)}
% 	\label{fig:Figuras_Tree_silhouettes-bitmap}
% \end{figure}

% \begin{figure}[htbp]
% 	\centering
% 	\includegraphics[height=1in]{snowman-vectorial}
% 	\includegraphics[height=3in]{snowman-vectorial}
% 	\includegraphics[height=6in]{snowman-vectorial}
% 	\caption{Imagem em formato PDF vectorial}
% 	\label{fig:Figuras_Tree_silhouettes-vectorial}
% \end{figure}

% Para agregar várias figuras numa única… Poderá assim referenciar o conjunto~\ref{fig:figura-completa}, a priemira delas~\ref{fig:novelo} ou a segunda~\ref{fig:nuvem}.


% \begin{figure}[htbp]
% 	\centering
%     \subbottom[Novelo de lã] {%
% 		\label{fig:novelo}
% 		\includegraphics[height=1in]{knitting-vectorial}
%     }
% \qquad\qquad
%     \subbottom[Tempestade com neve] {%
% 		\label{fig:nuvem}
% 		\includegraphics[height=1in]{snowstorm-vectorial}
%     }
%   \caption{Exemplo de utilização de \emph{subbottom}}
%   \label{fig:figura-completa}
% \end{figure}


% Para incluir listagens de código no seu documento, deverá incluir o pacote \emph{listings} e depois usar o ambiente \emph{lstlisting}, como exemplificado na Listagem~\ref{lst:HelloWorld}.

% \lstset{language=Java, caption=Hello World, label=lst:HelloWorld}
% \begin{lstlisting}
% /** 
%  * The HelloWorldApp class implements an application that
%  * simply prints "Hello World!" to standard output.
%  */
% class HelloWorldApp {%
%     public static void main(String[] args) {%
%         System.out.println("Hello World!"); // Display the string.
%     }
% }
% \end{lstlisting}

% \section{Equações}

% O LaTeX é uma ferramenta poderosa para escrever em estilo matemático. Permite inserir fórmulas no meio do texto como por exemplo esta: $ax^2 + bx + c = 0$. Também permite que as fórmulas sejam destacadas numa linha separada e centradas na página 
% $$x = \frac{-b \pm \sqrt{b^2-4ac}}{2a}$$
% \[x = \frac{-b \pm \sqrt{b^2-4ac}}{2a}\]
% ou numeradas 
% \begin{equation}
% aaa
% \label{eq:1}
% \end{equation}
% que depois pode ser referida no texto como sendo a equação~\ref{eq:1}
% $$\begin{array}{l}
% aa
% \end{array}
% $$

% \begin{eqnarray}
% a\\
% b\\
% c\\
% \end{eqnarray}
