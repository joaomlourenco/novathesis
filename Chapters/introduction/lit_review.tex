%!TEX root = ../../thesis_rui_almeida.tex


\section{Literature review}%
\label{sec:literature_review}

\subsection{Tomography}%
\label{sub:tomography}

Tomography is the cross-sectional imaging of an object through the use
of transmitted or reflected waves, captured by the object exposure to
the waves from a set of known angles. It has many different applications
in science, industry, and most prominently, medicine\todo{citatoins:
examples of industrial applications of tomography}. Since the invention
of the \gls{CT} machine in 1972, by Hounsfield~\cite{Gunderman2006},
tomographic imaging techniques have had a revolutionary impact, allowing
doctors to see inside their patients, without having to subject them to
more invasive procedures~\cite{Kak2001}.

Mathematical basis for tomography were set by Johannes Radon in 1917. At
the time, he postulated that  it is possible to represent a function
written in $\mathbb{R}$ in the space of straight lines, $\mathbb{L}$
through the function's line integrals. A line integral is an integral in
which the function that is being integrated is evaluated along a curved
path, a line. In the tomographic case, these line integrals represent a
measurement on a ray that traverses the \gls{ROI}.  Each set of line
integrals, characterized by an incidence angle, is called a projection
(see Figure~\ref{fig:projection}). To perform a tomographic
reconstruction, the machine must take many projections around the
object. To the set of projections arranged in matrix form by detector
and projection angle, we call sinogram. All reconstruction methods,
analytical and iterative, revolve around going from reality to sinogram
to image~\cite{Bruyant2002, Kak2001, Herman1973, Herman1995, Herman2009,
Defrise2003}.

\begin{figure}[htpb]
    \centering
    \includegraphics[width=0.5\textwidth]{img/png/projections.png}
    \caption{A schematic representation of a projection acquisition. In
    this image, taken from ~\cite{Herman2009}, the clear line that comes
    down at a diagonal angle is a projection.}
    \label{fig:projection}
\end{figure}

There are two broad algorithm families when it comes to tomographic
reconstruction, regarding the physics of the problem. It can involve
either non-diffracting sources (light travels in straight lines), such
as the X-Rays in a conventional \gls{CT} exam; or diffracting sources,
such as micro-waves or ultrasound in more research-oriented
applications~\cite{Kak2001}. In this document, I will not address the
latter family, since I will not be applying them in my work.

In any tomographic procedure, the first step is to gather information
from the target object. The first concept one requires for this is to
determine the problem's geometry. There are many different possible
geometry, however, there are two that are more important for this
thesis: parallel and fan-beam geometries. In the parallel case, there
are as many light sources as there are detectors. Light travels between
the source and the detector in straight lines, and the whole set rotates
around the object's location. Fan-beam geometries are characterized by
having only one light source which rotates around the target object. In
this geometry, a set of detectors are placed on the other side of the
object, and the lines (rays) between the source and the detectors
describe a fan, thus the name of the technique~\cite{Herman2009,
Kak2001}.

As far as algorithms are concerned, there are two main types: analytical
and iterative. The first family includes the most famous algorithm for
these applications, the \gls{FBP}\todo{Reference to the chapter in which
this technique is explained}. Iterative algorithms work by iteratively
searching for a solution to the reconstruction equation, which is
basically an underdetermined system of equations~\todo{check this!}.
There are numerous algorithms that work in this way, but in my work, I
have identified three that are extensively used, both in the field of
atmospheric tomography and in medicine: \gls{ART}, \gls{SART}, and
\gls{MLEM}. The simulator that was developed as part of this project
uses the last two and \gls{FBP}.

\subsection{\gls{DOAS}}%
\label{sub:doas}

Since the beginning of the 20\textsuperscript{th} century, scientists
have been using spectroscopy to measure reactive trace gases in the
atmosphere, especially ozone. The basis for these applications were set
by Bouguer, Lambert and Beer, which have separately presented the law
(Lambert-Beer's) that determines the relationship between light
extinction and the concentration of an absorber, when it must traverse a
medium in which this absorber is present. \gls{DOAS} is one of the
methods that is applied for this purpose. It was developed in 1976, by
Perner and his colleagues~\cite{Perner1976}, to detect and quantify the
hydroxyl radical in the atmosphere. The book by Jochen Stütz and Ulrich
Platt~\cite{Platt2007} is considered by most researchers one of the most
important references in the field and is present in most bibliographies
of the literature in this subject (it is also one of the main references
in this thesis). Platt, in particular, has been working with the
technique since its beginning, as one of the elements of Perner's team
that published the article about the hydroxyl radical mentioned some
lines above~\cite{Perner1976}.

Besides \gls{DOAS}, Lambert-Beer's law is the basis of many quantitative
spectroscopy applications. However, most of these techniques are used in
a laboratory context, in which conditions are controlled and very well
known. Atmospheric studies do not have this luxury. In the open
atmosphere, there are a number of factors, like Mie and Rayleigh
scattering, atmospheric turbulence or thermal fluctuations in the
optical path that make outdoor spectral measurements more complicated.
\gls{DOAS} is able to circumvent these difficulties by measuring
differential absorptions, which is to say the difference in absorption
between two different wavelengths~\cite{Platt2007, Merlaud2013}.

There are two modalities for \gls{DOAS} experiments, Active and Passive,
which differ mainly on the use of artificial or natural light sources,
respectively. Both methods have their advantages and disadvantages.
Active systems are more similar to a bench spectroscopy experiment.
Conditions are more controlled (starting with the light source) and
therefore, results are usually more reliable and precise, not to mention
simpler to reach, since there is no need to account for complex physical
phenomena, like radiative transfer~\cite{Platt2007}. However, these
systems do require additional material, and many times entire
infrastructures have to be built around them~\cite{Pundt2005}. Passive
systems, on the other hand, can be comprised of just a computer, a
spectrometer and a telescope, making them instrumentally much simpler
than their active counterparts. This flexibility also comes with the
possibility to develop new interesting sub-techniques, like
\gls{maxdoas}, which allows (for instance) the determination of the
stratospheric contribution of a certain trace gas (in opposition to its
tropospheric contribution). The mathematical processing of the acquired
spectra, nonetheless, is much more complex.

\subsection{DOAS Tomography}%
\label{sub:doas_tomography}

DOAS tomography is a relatively new subject within the realm of
\gls{DOAS}. It consists in the application of tomographic methods to
reconstruct a two-dimensional or three-dimensional \emph{map} of the
concentrations of trace gases in study. The seminal paper that
originated this and other remote sensing tomographic techniques was
published by Shepp in xxxx~\todo{This is written in some of the
tomographic DOAS papers.}. It is not by any means a very populated
literary space. In a systematic review that was performed as part of a
course I took during the development of this PhD thesis~\todo{Reference
the section in which this paper is more thoroughly discussed}, I managed
to identify a total of 13 papers that were clearly about DOAS
tomography. In doing this review, I have alaso found that the largest
DOAS tomography study was performed in Germany, during the first years
of the 21\textsuperscript{st} century. This research campagin, called
BAB-II~\cite{Laepple2004}, aimed to measure and map traffic-related
concentrations for \gls{no2} in the motorway that goes between
Heidelberg and Mannheim. A more recent effort that is mention worthy is
the paper by the Stutz~\cite{Stutz2016}, in which the group created a
tomographic system that was able to perform as a fence  line monitor for
a refinery in Houston, Texas. In between the two studies, and right
after BAB-II, Erna Frins published her paper~\cite{Frins2006}, in which
she described the use of a \gls{maxdoas} system alternately pointed
towards sun-illuminated and dark targets in a tomographic manner.

Besides the birds eye view of the literary panorama of the DOAS
tomography field, this systematic review allowed me to understand two
important gaps in the technology being employed for this research. The
first is that all of the studies that were featured in my review used a
very low number of tomographic projections (some dozens of line
integrals). This is a problem because it is the single most important
factor for the resolution of any tomographic procedure, and although
resolution is not an absolutely critical factor in atmospheric analysis
(because the sizes of the target objects - gas plumes - are very large
and diffuse), it is an important system feature that should be improved.
Moreover, the second important pattern that was clear from my research
was that all but one of the described systems were fixed, and the one
that was mobile was composed of a minimum of two spectral acquisition
devices and had one of the lowest numbers of projections in all papers.
This is a very important gap. DOAS tomography has the ability not only
to measure but also to map pollutant concentrations and can be an
invaluable technique in the fight to understand and track the movement
of pollutant plumes, but the fact that the available systems require
dedicated infrastructure to operate and have no mobility at all may be
the most important factor leading to the almost non-existing investment
DOAS tomography systems.

More than half of the world's population is expected to be living in
cities by the year 2050, and in the next 10 years we will see an
increase in the number of so-called megacities of more than
30\%~\cite{CABI2019}. This puts a lot of pressure on governmental
agencies and municipalities (specially in the West) to "smarten up"
their urban infrastructures, so that cities can harbor their inhabitants
with reasonable quality of life for everyone. In fact, a recent report
by xxx concluded that in the next yyy years, states and cities will
spend more than 150 billion Euros on this kind of
infrastructure~\todo{Check these numbers}. The flexibility and mobility
that the system I am developing brings to the table are two very heavy
points in its favor as a pollution mapping tool which is unobtainable
with the traditional methods, whether \emph{in-situ} or remote.
