%!TEX root = ../template.tex
%%%%%%%%%%%%%%%%%%%%%%%%%%%%%%%%%%%%%%%%%%%%%%%%%%%%%%%%%%%%%%%%%%%%
%% abstrac-de.tex
%% NOVA thesis document file
%%
%% Abstract in English
%%%%%%%%%%%%%%%%%%%%%%%%%%%%%%%%%%%%%%%%%%%%%%%%%%%%%%%%%%%%%%%%%%%%

\typeout{NT FILE dg_preface-de.tex}%

Tiere passen ihre Entscheidungen bei der Nahrungssuche an die Verfügbarkeit von Nährstoffen und ihre eigenen Bedürfnissen an. Diese Entscheidungen beruhen auf komplexen Berechnungen, um zu entscheiden, ob eine Nahrungsquelle ausgebeutet oder die Erkundung potenziell besserer Optionen fortgesetzt werden soll. Die neuronale Schaltkreislogik, die den dynamischen Berechnungen zur Abwägung von Erkundung und Ausbeutung während der Nahrungssuche zugrunde liegt, und die Art und Weise, wie diese durch innere Zustände beeinflusst werden, sind jedoch haupts\"achlich unbekannt.

\indent In dieser Arbeit stelle ich einen methodischen Ansatz vor, um zu untersuchen, wie die gemeine Fruchtfliege \textit{Drosophila melanogaster} entscheidet, wo und was sie frisst, um ihren aktuellen und potenziellen zukünftigen Nährstoffbedarf zu decken. Diese Methode kombiniert detaillierte und automatisierte Verhaltensquantifizierung mit einem groß angelegten neurogenetischen Manipulationsscreening, um neuronale Populationen zu identifizieren, die an verschiedenen Aspekten der Entscheidungsfindung der Fliegen während der Nahrungssuche beteiligt sind. Konkret haben wir einen Versuchsaufbau namens \textit{optoPAD} als optogenetisches System mit geschlossenem Regelkreis entwickelt, um neuronale Schaltkreise zu untersuchen, die am Fressverhalten beteiligt sind. Darüber hinaus haben wir das Paradigma der Nahrungswahl auf eine Arena mit mehreren Hefe- und Zuckerfeldern ausgeweitet, um die von den Fliegen bei der Erkundung der Arena getroffenen Entscheidungen zu untersuchen. Mit diesem Aufbau haben wir ein groß angelegtes neurogenetisches Verhaltensscreening durchgeführt, um die Auswirkungen des Ausschaltens der Aktivität in verschiedenen neuronalen Untergruppen auf Aspekte der Nahrungssuche zu testen. Wir entdeckten eine Neuronenpopulation, die in eine zentrale Region des Insektengehirns projiziert, die für die Entscheidung, an einem Futterplatz anzuhalten, wichtig ist. Wir konnten zeigen, dass ein einziges Neuronenpaar innerhalb dieser Population für den Erkundungstrieb der Fliege erforderlich ist und dass die Aktivität dieses Neurons durch den Proteinstatus der Fliege moduliert wird. Abschließend stelle ich die Schlussfolgerungen dieser Studie vor, um einen integrativen theoretischen Rahmen für die Entscheidungsfindung der Fliegen bei der Nahrungssuche mit dem normativen Ziel einer ausgewogenen Nährstoffaufnahme zu schaffen.

\indent Unsere Ergebnisse offenbaren ein neuronales Substrat, das für die Regulierung komplexer, ethologisch relevanter Entscheidungen bei der Nahrungssuche wichtig ist. Dies ist ein entscheidender Schritt hin zu einer mechanistischen Erklärung der kognitiven Funktionen, die für die Berechnung des Kompromisses zwischen Erkundung und Ausbeutung erforderlich sind, und damit auch für die Abstimmung der Ergebnisse der Nahrungssuche auf die physiologischen Bedürfnisse.

% Palavras-chave do resumo em Inglês
\keywords{
  Systemneurowissenschaften \and
  Verhalten \and
  Nahrungssuche \and
  Erkundung \and
  Neuronale Berechnungen\and
  Kognition
}
