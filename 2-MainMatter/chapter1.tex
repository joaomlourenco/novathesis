%!TEX root = ../template.tex
%%%%%%%%%%%%%%%%%%%%%%%%%%%%%%%%%%%%%%%%%%%%%%%%%%%%%%%%%%%%%%%%%%%
%% chapter1.tex
%% NOVA thesis document file
%%
%% Chapter with introduction
%%%%%%%%%%%%%%%%%%%%%%%%%%%%%%%%%%%%%%%%%%%%%%%%%%%%%%%%%%%%%%%%%%%

\typeout{NT FILE chapter1.tex}%

\chapter{Introduction}
\label{chap:introduction}

\section{Purpose and Scope}

The \gls{novathesis} template is a comprehensive \LaTeX{} class and project skeleton designed to simplify the preparation of academic theses and dissertations. Representing a synthesis of many years of practical experience in producing academic documents with \LaTeX{}, its objective is not only to provide a consistent visual identity but also to encourage good typesetting practices and reproducible workflows. It provides a robust, extensible, and highly configurable foundation for students, researchers, and supervisors who require professional-quality documents that conform to institutional standards. 

The primary goals of the \gls{novathesis} template are:
\begin{itemize}
  \item To ensure \textbf{consistency} across theses produced by different students and institutions;
  \item To provide \textbf{institutional presets} that reproduce the visual identity of each supported school;
  \item To maintain a clear \textbf{separation between content and style}, allowing users to focus on writing rather than formatting;
  \item To support \textbf{multi-language documents} with automatic translation of structural elements (titles, captions, lists, etc.);
  \item To integrate \textbf{modern bibliography management} through \texttt{biblatex} and \texttt{biber};
  \item To offer a \textbf{modular and automated build system} through \texttt{make} and \texttt{latexmk} (and a \texttt{build.py} Python script).
\end{itemize}

This manual explains how to install, configure, and use the \gls{novathesis} template.  
It is intended for users with a basic knowledge of \LaTeX{}, but it also serves as a detailed reference for advanced customization and class maintenance.

\section{Design Philosophy}

The \gls{novathesis} system is built on three principles: \emph{modularity}, \emph{transparency}, and \emph{maintainability}.

\subsection{Modularity}

Every component of the template has a clearly defined role.  
Configuration files are isolated from the main document; each thematic area—fonts, bibliography, cover page, lists—has its own configuration module.  
This structure allows users to modify or extend a specific feature without affecting unrelated parts of the document.

\subsection{Transparency}

All configuration is done through plain-text files in the \texttt{0-Config/} directory.  
There are no hidden or hard-coded dependencies: every visual and logical choice is traceable and user-accessible.  
This transparency makes the system easy to audit, debug, and adapt to institutional requirements.

\subsection{Maintainability}

The template evolves with each new version of \LaTeX{} and the supported class packages.  
It favors standard packages, ensuring long-term compatibility and minimizing maintenance overhead.

\section{Supported Document Types}

\gls{novathesis} can produce various document types through a single configuration parameter.  
The \texttt{doctype} option automatically adjusts the document covers, headings, titles, and certain layout dimensions to each degree level and school.  

The supported document types include:

\begin{itemize}
  \item \texttt{phd} – Doctoral dissertation;
  \item \texttt{msc} – Master's thesis;
  \item \texttt{bsc} – Bachelor's dissertation.
\end{itemize}

\section{Institutional Support}

The \gls{novathesis} template supports multiple institutions and departments by means of preset configuration profiles.  
Each profile defines the official layout, logos, colors, and front-matter fields required by the corresponding institution.

As of version~7.6.0, the template includes presets for:

\begin{itemize}
  
  \item Universidade do Porto
  \begin{itemize}
    \item \texttt{uporto/fcup} – Faculdade de Ciências da Universidade do Porto;
  \end{itemize}

  \item Instituto Politécnico de Setúbal
  \begin{itemize}
    \item \texttt{ips/ests} – Escola Superior de Tecnologia de Setúbal;
  \end{itemize}

  \item Instituto Universitário de Lisboa
  \begin{itemize}
    \item \texttt{iscteiul/eta} – Escola de Tecnologia e Arquitectura;
  \end{itemize}
  
  \item \texttt{uminho} – University of Minho;
  \begin{itemize}
    \item \texttt{uminho/eaad} - Escola de Arquitetura, Arte e Design;
    \item \texttt{uminho/ec} - Escola de Ciências;
    \item \texttt{uminho/ed} - Escola de Direito;
    \item \texttt{uminho/eeng} - Escola de Engenharia;
    \item \texttt{uminho/elach} - Escola de Letras, Artes e Ciências Humanas​;
    \item \texttt{uminho/emed} - Escola de Medicina​;
    \item \texttt{uminho/epsi} - Escola de Psicologia;
    \item \texttt{uminho/ese} - Escola Superior de Enfermagem;
    \item \texttt{uminho/ics} - Instituto de Ciencias Sociais;
    \item \texttt{uminho/ie} - Instituto de Educação;
    \item \texttt{uminho/i3bs} - Instituto de Investigação em Biomateriais, Biodegradáveis e Biomiméticos​;
  \end{itemize}

  \item Universidade de Lisboa
  \begin{itemize}
    \item \texttt{ulisboa/ist} — Instituto Superior Técnico;
    \item \texttt{ulisboa/iseg} — Instituto Superior de Economia e Gestão;
    \item \texttt{ulisboa/fmv} — Faculdade de Medicina Veterinária;
    \item \texttt{ulisboa/fcul} — Faculdade de Ciências da Universidade de Lisboa;
  \end{itemize}

  \item Outras Instituições de Ensino Superior
  \begin{itemize}
    \item \texttt{other/esep} — Escola Superior de Enfermagem do Porto
  \end{itemize}

  \item Universidade NOVA de Lisboa
  \begin{itemize}
    \item \texttt{nova/itqb} - Instituto de Tecnologia Química e Biológica António Xavier;
    \item \texttt{nova/ims} – NOVA Information Management School;
    \item \texttt{nova/fct} – Faculdade de Ciências e Tecnologia;
    \item \texttt{nova/ensp} - Escola Nacional de Saúde Pública;
    \item \texttt{nova/fcsh} - Faculdade de Ciências Sociais e Humanas;
  \end{itemize}
  
  \item Universidade de Lisboa
  \begin{itemize}
    \item \texttt{ulisboa/fmv} – Faculty of Veterinary Medicine;
    \item \texttt{ulisboa/iseg} – School of Economics and Management;
    \item \texttt{ulisboa/ist} – Instituto Superior Técnico;
  \end{itemize}
  
\end{itemize}

Institutional presets are easily activated by uncommenting and editing a single configuration line in \texttt{0-Config/1\_novathesis.tex}:
\begin{verbatim}
\ntsetup{school=nova/fct}
\end{verbatim}

Advanced users may also define their own institution by creating a configuration files under \texttt{NOVAthesisFiles/Schools/UNIVERSITY/SCHOOL} and referencing \texttt{UNIVERSITY/SCHOOL} in the setup.

\section{Project Layout Overview}

The \gls{novathesis} project is organized to separate configuration, content, and auxiliary resources into dedicated directories.  
A simplified view of the structure is shown below:

\begin{verbatim}
_
|—— 0-Config/         Configuration files
|—— 1-FrontMatter/    Abstracts, acknowledgments, acronyms, etc.
|—— 2-MainMatter/     Main text chapters
|—— 3-BackMatter/     Appendices and Annexes
|—— 4-Bibliography/   BibTeX/BibLaTeX databases
|—— 5-Figures/        Image resources for figures
|—— template.tex      Main document entry point
|—— novathesis.cls    Class definition
|—— NOVAthesisFiles/  Internal class files and strings
|—— Makefile          Automated build system
\end{verbatim}

Users are encouraged to keep the directory layout unchanged, as some internal macros rely on relative paths for file inclusion.

\section{Conventions Used in This Manual}

Throughout this manual, filenames and paths appear in \texttt{typewriter font}, \LaTeX{} macros in \verb|\verb| notation, and code examples are presented in shaded blocks.  
When an option or parameter is introduced, its default value is indicated in parentheses.

For example:
\begin{quote}
  \verb|\ntsetup{doctype=msc}| \hfill (default: \texttt{msc})
\end{quote}

Optional configuration values are denoted as:
\begin{quote}
  \verb|\ntsetup{school=<identifier>}|, where \texttt{<identifier>} may be one of
  \texttt{nova/fct}, \texttt{ulisboa/ist}, etc.
\end{quote}

\section{Target Audience}

This manual addresses:
\begin{itemize}
  \item \textbf{Students} preparing academic theses or dissertations;
  \item \textbf{Supervisors} and \textbf{faculty} defining institutional guidelines;
  \item \textbf{Template maintainers} who need to extend or adapt the class;
  \item \textbf{System administrators} deploying \LaTeX{} environments for academic use.
\end{itemize}

It assumes a basic working knowledge of \LaTeX{}, including compilation, referencing, and package usage.

\section{Reading Guide}

Chapters~\ref{chap:installation} and~\ref{chap:getting-started} provide step-by-step instructions for installation and first compilation.  
Subsequent chapters describe the configuration system, document structure, and build tools in depth.  
Advanced topics—such as creating new institutional profiles, customizing the cover, or defining new font themes—are discussed later in the manual.

Readers who only wish to produce a thesis using a supported preset may safely skip the technical sections and focus on Chapters~\ref{chap:getting-started} through~\ref{chap:document-structure}.

