%!TEX root = ../template.tex
%%%%%%%%%%%%%%%%%%%%%%%%%%%%%%%%%%%%%%%%%%%%%%%%%%%%%%%%%%%%%%%%%%%%
%% chapter2.tex
%% NOVA thesis document file
%%
%% Chapter with the template manual
%%%%%%%%%%%%%%%%%%%%%%%%%%%%%%%%%%%%%%%%%%%%%%%%%%%%%%%%%%%%%%%%%%%%

\typeout{NT FILE chapter2.tex}%

\chapter{Installation}
\label{chap:installation}

\section{Overview}

This chapter describes how to install and prepare the \gls{novathesis} template for use on local systems and in cloud environments.  
The template is compatible with all major \LaTeX{} distributions that support \texttt{latexmk}, \texttt{biber}, and the modern \texttt{memoir} class infrastructure.

\gls{novathesis} may be used on:
\begin{itemize}
  \item \textbf{macOS} systems through the \texttt{MacTeX} distribution;
  \item \textbf{GNU/Linux} systems through the \texttt{TeX\,Live} distribution;
  \item \textbf{Windows} systems through the \texttt{MiKTeX} distribution;
  \item \textbf{Cloud environments} such as Overleaf (subject to certain limitations).
\end{itemize}

The following sections provide detailed installation procedures for each platform, followed by general requirements and post-installation checks.

\section{System Requirements}

To compile documents using the \gls{novathesis} class, the following components are required:

\begin{itemize}
  \item A complete \LaTeX{} distribution (MacTeX, TeX\,Live, or MiKTeX) released in or after~2022;
  \item The auxiliary build tools \texttt{latexmk} and \texttt{make};
  \item A bibliography processor (\texttt{biber}, preferred, or \texttt{bibtex});
  \item A modern PDF viewer (such as \texttt{Preview}, \texttt{Evince}, or \texttt{SumatraPDF});
  \item For XeLaTeX or LuaLaTeX compilation, access to OpenType or system fonts.
\end{itemize}

The \gls{novathesis} class does not depend on any non-standard binary tools or external scripts.  
All operations can be performed within a standard \LaTeX{} installation.

\section{Installation on macOS (MacTeX)}

\subsection{Step 1: Install MacTeX}

MacTeX provides a complete and self-contained \LaTeX{} environment for macOS.  
It can be obtained from:

\begin{quote}
  \url{https://tug.org/mactex/}
\end{quote}

Download and install the \texttt{MacTeX.pkg} package (approximately 5~GB).  
This installation includes \texttt{TeX\,Live}, \texttt{latexmk}, \texttt{biber}, and common front-ends such as TeXShop.

\subsection{Step 2: Verify the Installation}

After installation, open the Terminal and run:

\begin{verbatim}
which pdflatex
which latexmk
which biber
\end{verbatim}

If these commands return valid paths (usually under \texttt{/Library/TeX/texbin}), the system is correctly configured.  
You may also verify the distribution version:

\begin{verbatim}
pdflatex --version
biber --version
\end{verbatim}

\subsection{Step 3: Obtain the \gls{novathesis} Template}

Clone or extract the template into a working directory of your choice, for example:

\begin{verbatim}
cd ~/Documents/Thesis
git clone https://github.com/<your-org>/novathesis.git
\end{verbatim}

Alternatively, download and extract the ZIP archive.

\subsection{Step 4: Compile a Test Document}

Within the project directory, run:

\begin{verbatim}
make xe
\end{verbatim}

or equivalently:

\begin{verbatim}
latexmk -xelatex template.tex
\end{verbatim}

If compilation completes without error, the environment is operational.  
The generated \texttt{template.pdf} will appear in the same directory.

\section{Installation on GNU/Linux (TeX\,Live)}

\subsection{Step 1: Install TeX\,Live}

Most Linux distributions provide \texttt{TeX\,Live} via their package manager.  
For a full installation, execute one of the following commands:

\begin{itemize}
  \item \textbf{Debian/Ubuntu:}
  \begin{verbatim}
  sudo apt install texlive-full
  \end{verbatim}

  \item \textbf{Fedora:}
  \begin{verbatim}
  sudo dnf install texlive-scheme-full
  \end{verbatim}

  \item \textbf{Arch Linux:}
  \begin{verbatim}
  sudo pacman -S texlive-most biber
  \end{verbatim}
\end{itemize}

The \texttt{texlive-full} or equivalent meta-package ensures that all dependencies used by \gls{novathesis} are available.

\subsection{Step 2: Verify the Installation}

Check the presence of required tools:

\begin{verbatim}
which pdflatex
which biber
which latexmk
\end{verbatim}

All should return executable paths. If any are missing, install them individually using the package manager.

\subsection{Step 3: Obtain and Compile}

Download or clone the template, then compile as follows:

\begin{verbatim}
git clone https://github.com/<your-org>/novathesis.git
cd novathesis
make pdf     # or make xe / make lua
\end{verbatim}

When using XeLaTeX or LuaLaTeX, ensure that system fonts referenced in your chosen font theme are installed on the machine.

\section{Installation on Windows (MiKTeX)}

\subsection{Step 1: Install MiKTeX}

MiKTeX provides an integrated package manager and is well suited for Windows systems.  
Download the installer from:

\begin{quote}
  \url{https://miktex.org/download}
\end{quote}

Run the installer with administrative privileges and select \emph{Install missing packages on-the-fly: Yes}.

\subsection{Step 2: Verify the Installation}

Open the \textbf{MiKTeX Console} and ensure that the following packages are installed:
\begin{quote}
\texttt{latexmk}, \texttt{biber}, \texttt{biblatex}, \texttt{memoir}.
\end{quote}

Alternatively, from the Command Prompt:

\begin{verbatim}
latexmk --version
biber --version
\end{verbatim}

\subsection{Step 3: Obtain the Template}

Download the project ZIP or clone it using Git:

\begin{verbatim}
git clone https://github.com/<your-org>/novathesis.git
cd novathesis
\end{verbatim}

\subsection{Step 4: Compile the Template}

Run the following in the Windows Command Prompt or PowerShell:

\begin{verbatim}
latexmk -pdf template.tex
\end{verbatim}

or

\begin{verbatim}
make pdf
\end{verbatim}

If the command \texttt{make} is not available, it may be installed through \texttt{Git Bash} or \texttt{MinGW}.  
The produced \texttt{template.pdf} should appear in the same directory upon successful compilation.

\section{Installation in Overleaf (Cloud)}

\subsection{Overview}

The \gls{novathesis} template can also be compiled within \textbf{Overleaf}, an online \LaTeX{} editing environment.  
However, due to its size, modularity, and dependence on \texttt{latexmk} and \texttt{biber}, full compilation of \gls{novathesis} requires an \textbf{Overleaf Professional or Group Plan}.  
Free-tier accounts do not provide sufficient memory or compilation time to complete the build.

\subsection{Step 1: Create a New Project}

1. Log in to Overleaf at \url{https://www.overleaf.com}.  
2. Create a new blank project.  
3. Upload the entire \gls{novathesis} directory structure (including subfolders).  
   This may be done via drag-and-drop or by uploading a ZIP archive.

\subsection{Step 2: Configure the Compiler}

Open the \textbf{Menu} (top left) and set the compiler to:
\begin{quote}
\texttt{XeLaTeX} \quad or \quad \texttt{LuaLaTeX}
\end{quote}
depending on your selected font configuration.  
Ensure that the bibliography tool is set to \texttt{Biber}.

\subsection{Step 3: Compile}

Click \textbf{Recompile}.  
The process may take several minutes depending on server load and account limits.  
If compilation fails with memory or time errors, consider building locally or upgrading the Overleaf account.

\subsection{Step 4: Managing Large Projects}

Due to file system limitations, it is recommended to:
\begin{itemize}
  \item Avoid uploading unnecessary figures or datasets;
  \item Use compressed image formats (\texttt{.pdf}, \texttt{.png});
  \item Use Overleaf’s \emph{Git integration} to synchronize changes with a local copy.
\end{itemize}

\section{Post-installation Validation}

After installation and initial compilation, verify the following:

\begin{enumerate}
  \item The file \texttt{template.pdf} was generated without errors;
  \item The title page and metadata match your configuration in \texttt{0-Config/1\_novathesis.tex};
  \item Bibliography entries from \texttt{4-Bibliography/bibliography.bib} appear correctly;
  \item Figures included from \texttt{5-Figures/} display as expected;
  \item Language settings (captions, lists, chapter titles) match the selected \texttt{mainlanguage}.
\end{enumerate}

Once validated, the environment is ready for content development.

\section{Maintenance and Updates}

The \gls{novathesis} template may evolve with newer releases of \LaTeX{} or institutional requirements.  
To update an existing installation:

\begin{itemize}
  \item If using Git, pull the latest changes:
  \begin{verbatim}
  git pull origin main
  \end{verbatim}
  \item If using a ZIP distribution, download the new version and replace the class and configuration files as needed.
\end{itemize}

Always back up your \texttt{0-Config} and content folders before applying updates.

\section{Summary}

This chapter outlined the procedures for installing \gls{novathesis} across all supported environments.  
Local installations are recommended for large or complex projects, as they offer full control and faster compilation times.  
Cloud-based compilation through Overleaf remains convenient for collaboration, provided that a Professional or institutional account is used.

Subsequent chapters describe how to configure, customize, and extend the template for your specific academic institution and degree requirements.
