%!TEX root = ../template.tex
%%%%%%%%%%%%%%%%%%%%%%%%%%%%%%%%%%%%%%%%%%%%%%%%%%%%%%%%%%%%%%%%%%%%
%% chapter4.tex
%% NOVA thesis document file
%%
%% Chapter with lots of dummy text
%%%%%%%%%%%%%%%%%%%%%%%%%%%%%%%%%%%%%%%%%%%%%%%%%%%%%%%%%%%%%%%%%%%%

\typeout{NT FILE chapter4.tex}%

\chapter{Configuration}
\label{chap:configuration}

\section{Overview}

This chapter documents the \gls{novathesis} configuration subsystem as shipped in version~7.6.0.  
All user-facing configuration is centralized under \texttt{0-Config/}. The design separates \emph{global class behaviour} from \emph{project content} and \emph{institutional presets}. Users should edit only the files in \texttt{0-Config/} and content folders; the class file \texttt{novathesis.cls} and internal assets under \texttt{\gls{novathesis}Files/} must remain unchanged.

\begin{flushleft}
\textbf{Configuration files (as shipped):}
\end{flushleft}

\begin{verbatim}
0-Config/
  0_memoir.tex      % Memoir-class options (low-level layout toggles)
  1_novathesis.tex  % Main \gls{novathesis} options (doctype, language, media, styles…)
  2_biblatex.tex    % Bibliography backend and style
  3_cover.tex       % Cover/front-matter metadata and committee
  4_files.tex       % List of included files: abstracts, chapters, appendices…
  5_packages.tex    % Extra (user) packages and demo setup
  6_list_of.tex     % Which lists to print (figures/tables/algorithms/listings…)
  9_*.tex           % Institutional presets (FCT NOVA, IMS, IST, ISEG, FMV, UMinho, UPorto/FC)
\end{verbatim}

Unless stated otherwise, options are applied via the key–value interface
\verb|\ntsetup{<key>=<value>}|.
Bibliography options use \verb|\ntbibsetup{…}|.
Language-string overrides use \verb|\ntlangsetup{…}|.

\paragraph{Editing policy.}
\begin{itemize}
  \item Keep your changes \emph{in place} inside \texttt{0-Config/}. Do not modify \texttt{novathesis.cls}.
  \item Prefer uncommenting documented options over inventing new macros.
  \item When in doubt, search for the option name in the shipped \texttt{.tex} files; most defaults are indicated in comments.
\end{itemize}

\section{\texorpdfstring{\texttt{0\_memoir.tex}}{0\_memoir.tex}: Memoir Options}
\label{sec:cfg-memoir}

\texttt{0\_memoir.tex} is reserved for options supported by the \texttt{memoir} class (page layout, trims, headers/footers, etc.). Use this file if you need low-level control beyond \gls{novathesis}’ high-level keys.

\begin{itemize}
  \item Place any \texttt{memoir} options or setup here (see the \textit{memman} manual).
  \item Institution-specific stubs may appear here (e.g., final classification or funding notes for some presets). Keep these commented unless applicable.
\end{itemize}

\section{\texorpdfstring{\texttt{1\_novathesis.tex}}{1\_novathesis.tex}: Main Options}
\label{sec:cfg-main}

This is the primary configuration file. It exposes the document’s global behaviour and presentation. The file is extensively commented; below we summarize the keys as they appear in the shipped template.

\subsection*{Document identity and lifecycle}
\begin{description}
  \item[\texttt{doctype}] The document kind.
  \begin{flushleft}
  \verb|\ntsetup{doctype=<value>}|\quad where \verb|<value>| is one of:
  \verb|phd|, \verb|phdplan|, \verb|phdprop|, \verb|msc|, \verb|mscplan|,
  \verb|bsc|, \verb|plain|.
  \end{flushleft}

  \item[\texttt{school}] Institutional preset identifier (activates \texttt{0-Config/9\_<…>.tex} and internal assets).\par
  Example identifiers as shipped (non-exhaustive; see comments in the file):
  \begin{quote}
  \small
  \verb|nova/fct|, \verb|nova/fct/di-adc|, \verb|nova/fct/cbbi|, \verb|nova/fct/blue|, \verb|nova/fct/green|, \verb|nova/fct/brown|, \verb|nova/fct/red|, \verb|nova/itqb/gray|, \verb|nova/itqb/green|, \verb|nova/fcsh|, \verb|nova/ensp|,\newline
  \verb|nova/ims|, \verb|nova/ims/csig|, \verb|nova/ims/ddm|, \verb|nova/ims/dsaa|, \verb|nova/ims/egi|, \verb|nova/ims/gi|, \verb|nova/ims/gt|,\newline
  \verb|ulisboa/ist|, \verb|ulisboa/fc|, \verb|ulisboa/fmv|, \verb|ulisboa/iseg|,\newline
  \verb|uminho/ead|, \verb|uminho/ese|, \verb|uminho/eeng|, \verb|uminho/elach|, \verb|uminho/ed|, \verb|uminho/ec|, \verb|uminho/i3bs|, \verb|uminho/emed|, \verb|uminho/ie|, \verb|uminho/ics|, \verb|uminho/epsi|, \verb|uminho/eeg|,\newline
  \verb|iscteiul/eta|, \verb|ips/ests|, \verb|ipl/isel|, \verb|ipl/isel/meb|,\newline
  \verb|ulht/deisi|, \verb|ulht/mge|, \verb|uporto/fc|, \verb|other/esep|.
  \end{quote}

  \item[\texttt{docstatus}] Working stage of the document:
  \verb|working| (reduced front matter), \verb|provisional| (submission, no committee), \verb|final| (final, committee included).
\end{description}

\subsection*{Languages and abstracts}
\begin{description}
  \item[\texttt{lang}] Main document language (ISO-639-1 two-letter): \verb|en|, \verb|pt|, \verb|fr|, \verb|it|, \verb|de|, \verb|es|, \verb|gr| (Greek), \verb|uk| (Ukrainian).
  \item[\texttt{abstractorder}] Controls which abstracts print and in which order.  
  Examples:
\begin{verbatim}
\ntsetup{abstractorder={en,pt,uk,gr}}
\ntsetup{abstractorder={pt={en,pt,fr}}}
\end{verbatim}
  Default for \texttt{en}: \texttt{en,pt}.  
  For language \texttt{L}: \texttt{L,en}.
  \item[\texttt{lang/extra}] Additional languages used in the document (beyond those in abstracts), e.g.,
\begin{verbatim}
\ntsetup{lang/extra={de,es}}
\end{verbatim}
\end{description}

\subsection*{Media and pagination}
\begin{description}
  \item[\texttt{media}] \verb|screen| (equal margins, coloured links) or \verb|paper| (bookish margins, black links).
  \item[\texttt{numberallpages}] Whether to number all pages (except the cover); default \texttt{false}.
  \item[\texttt{print/index}] Print a word index at the end; default \texttt{false}.
  \item[\texttt{print/timestamp}] Print the PDF build timestamp on the cover; default \texttt{true}.
\end{description}

\subsection*{Glossaries and indices}
\begin{description}
  \item[\texttt{gnumberlist}] Whether glossary entries list the pages on which they occur (reverse-index behaviour); default \texttt{true}.
\end{description}

\subsection*{Typography and style}
\begin{description}
  \item[\texttt{style/url}] Use the main text font for URLs; default = the \texttt{hyperref} default (set explicitly to \verb|same| to match body font).
  \item[\texttt{style/font}] Font theme. Shipped style modules under \texttt{\gls{novathesis}Files/FontStyles/}:
  \begin{quote}
  \small
  \verb|bookman|, \verb|erewhon|, \verb|libertine|, \verb|newpx|, \verb|opensans|, \verb|scholax|,\newline
  \verb|arial|*, \verb|calibri|*, \verb|newsgott|*, \verb|kieranhealy|*, \verb|futura|*.
  \end{quote}
  The styles marked with (*) require XeLaTeX/LuaLaTeX (system/OpenType fonts).
\end{description}

\subsection*{Book spine and debugging}
\begin{description}
  \item[\texttt{spine/layout}] \verb|no| (do not print), \verb|full| (full page), \verb|trim| (print and crop to spine width). Default \verb|trim| for \texttt{docstatus=final}, otherwise \verb|no|.
  \item[\texttt{spine/width}] Force a specific spine width (e.g., \verb|2cm|). Default is an automatic width assuming double-sided 80\,g/m\textsuperscript{2} paper.
  \item[\texttt{debug}] \verb|{cover,spine}| prints cover grids and spine markers for development.
\end{description}

\section{\texorpdfstring{\texttt{2\_biblatex.tex}}{2\_biblatex.tex}: Bibliography}
\label{sec:cfg-biblatex}

This file configures the bibliography backend and common styles via \verb|\ntbibsetup|. The shipped defaults favour \texttt{biblatex}+\texttt{biber}.

\begin{description}
  \item[\texttt{backend}] \verb|biber| (default) or \verb|bibtex|.
  \item[\texttt{style}] Exemplos provided (uncomment exactly one):
\begin{verbatim}
% Numeric styles:
% \ntbibsetup{style=numeric-comp, sortcites=true, sorting=none}
% Alphabetic style:
% \ntbibsetup{style=alphabetic, sortcites=true, sorting=nyt}
% Author–year (compact):
% \ntbibsetup{style=authoryear-comp, sortcites=true, sorting=nyt}
% APA-like (natbib compatibility):
% \ntbibsetup{backend=biber,natbib=true,style=apa}
\end{verbatim}
\end{description}

\paragraph{Notes.}
\begin{itemize}
  \item Use \verb|\textcite| for “Author (year)” and \verb|\parencite| for “(Author, year)”.
  \item When switching backends, \emph{clean} the build (\texttt{make clean}) and recompile.
\end{itemize}

\section{\texorpdfstring{\texttt{3\_cover.tex}}{3\_cover.tex}: Cover and Front Matter}
\label{sec:cfg-cover}

Cover metadata, degree names, optional specialization, sponsors, and committee are defined here using high-level macros.

\subsection*{Titles}
Use \verb|\nttitle(type,lang){text}| with:
\verb|type| \(\in\) \verb|{main, sub, spine}|\,, \verb|lang| \(\in\) available languages.

\begin{verbatim}
\nttitle(main,en){A Very Long and Impressive\\Thesis Title}
\nttitle(main,pt){Um Título de Tese Longo\\e Impressionante}
\nttitle(sub,en){Some thoughts on the Life, the Universe,\\and Everything Else}
% \nttitle(spine,en){A Very Long and Impressive Thesis Title} % (no manual line breaks)
\end{verbatim}

\subsection*{Degree, specialization, sponsors}
\begin{verbatim}
% Degree label (only defines if still undefined elsewhere):
\ntdegreename*(pt){Nome do Programa de Estudos}
\ntdegreename*(en){Study Program Name}

% Optional specialization (defines if undefined):
\ntspecialization*(pt){Designação da Especialidade}
\ntspecialization*(en){Specialization Name}

% Optional list of sponsors (per language):
% \ntsponsors(en){Company A, Grant X, ...}
% \ntsponsors(pt){Empresa A, Bolsa X, ...}
\end{verbatim}

\subsection*{Committee}
Add committee members with role and gender markers:
\begin{quote}
\small
\verb|\ntaddperson{committee}(<role>,<gender>){Name, Position, Institution}|
\end{quote}
Roles: \verb|c| (chair), \verb|r| (rapporteur), \verb|a| (adviser), \verb|ca| (co-adviser), \verb|m| (member), \verb|g| (guest).  
Gender: \verb|m| (male), \verb|f| (female).

\begin{verbatim}
\ntaddperson{committee}(c,m){Chair Name, Full Professor, FCT-NOVA}
\ntaddperson{committee}(r,m){Rapporteur Name, Associate Professor, Other Univ.}
\ntaddperson{committee}(m,f){Member Name, Assistant Professor, Another Univ.}
% \ntaddperson{committee}(a,m){Adviser present, Assoc. Prof., University}
% \ntaddperson{committee}(ca,f){Co-adviser present, Assoc. Prof., University}
% \ntaddperson{committee}(g,m){Guest member, Title, Institution}
\end{verbatim}

\paragraph{Institution-specific guidance.}
Some Schools request additional fields (e.g., sponsors, SDGs, exam date, embargo). See Section~\ref{sec:cfg-schools}.

\section{\texorpdfstring{\texttt{4\_files.tex}}{4\_files.tex}: File Inventory}
\label{sec:cfg-files}

This file enumerates the content files that compose your thesis. Use \verb|\ntaddfile| with the appropriate \emph{kind}:

\begin{description}
  \item[\texttt{bib}] Bibliography databases (\texttt{.bib}):
\begin{verbatim}
\ntaddfile{bib}{bibliography.bib}
% \ntaddfile{bib}{another.bib}
\end{verbatim}

  \item[\texttt{dedicatory}, \texttt{acknowledgements}, \texttt{quote}] Printed only for final document types (\texttt{bsc}, \texttt{msc}, \texttt{phd}):
\begin{verbatim}
\ntaddfile{dedicatory}{dedicatory}
\ntaddfile{acknowledgements}{acknowledgements}
\ntaddfile{quote}{quote}
\end{verbatim}

  \item[\texttt{abstract}[lang]] Abstracts in multiple languages:
\begin{verbatim}
\ntaddfile{abstract}[pt]{abstract-pt}
\ntaddfile{abstract}[en]{abstract-en}
% \ntaddfile{abstract}[de]{abstract-de}
% \ntaddfile{abstract}[es]{abstract-es}
% \ntaddfile{abstract}[fr]{abstract-fr}
\end{verbatim}

  \item[\texttt{chapter}] Chapters (optionally restricted to some doctypes):
\begin{verbatim}
\ntaddfile{chapter}{chapter1}
\ntaddfile{chapter}{chapter2}
\ntaddfile{chapter}{chapter3}
\ntaddfile{chapter}{chapter4}
% Examples:
% \ntaddfile{chapter}[phd]{my-chapter}
% \ntaddfile{chapter}[phd,phdplan]{proposal-background}
\end{verbatim}

  \item[\texttt{appendix}, \texttt{annex}] Back matter:
\begin{verbatim}
\ntaddfile{appendix}{appendix1}
\ntaddfile{appendix}{appendix2}
\ntaddfile{annex}{annex1}
\end{verbatim}

  \item[\texttt{cover}[slot]] User-defined covers (override standard ones): \verb|slot| \(\in\) \verb|1| (front), \verb|N| (back), \verb|spine|.
\begin{verbatim}
% \ntaddfile{cover}[1]{cover-front}
% \ntaddfile{cover}[N]{cover-back}
% \ntaddfile{cover}[spine]{cover-spine}
% Disable second cover if needed:
% \ntsetup{print/secondcover=false}
\end{verbatim}
\end{description}

\section{\texorpdfstring{\texttt{5\_packages.tex}}{5\_packages.tex}: Extra Packages}
\label{sec:cfg-packages}

This file is intended for user-level package additions and language-string overrides. The shipped file includes \emph{demo packages} for examples (e.g., \texttt{float}, \texttt{wrapfig}); remove these in production.

\paragraph{Language-string overrides.}
Use \verb|\ntlangsetup{<lang>/<key>=<value>}| to redefine localized strings (e.g., ToC name):

\begin{verbatim}
% \ntlangsetup{pt/contentsname=O MEU ÍNDICE}
% \ntlangsetup{en/contentsname=MY TABLE OF CONTENTS}
\end{verbatim}

\paragraph{Listings and minted.}
If you enable \texttt{listings} or \texttt{minted}, Section~\ref{sec:cfg-listof} shows how the corresponding “List of Listings” is enabled automatically.

\section{\texorpdfstring{\texttt{6\_list\_of.tex}}{6\_list\_of.tex}: Lists to Print}
\label{sec:cfg-listof}

Controls which \emph{lists} appear (List of Figures, Tables, Algorithms, Listings, etc.) using \verb|\ntaddlistof{<name>}|. The file also contains logic that adds certain lists automatically if the corresponding package is loaded.

\begin{verbatim}
\ntaddlistof{listoffigures}
\ntaddlistof{listoftables}
\end{verbatim}

Algorithm lists are added depending on the package in use (\texttt{algorithm2e}, \texttt{algorithms}, \texttt{algorithm}, or \texttt{coloredtheorem}).  
Listings are added for \texttt{listings} (\verb|lstlistoflistings|) or \texttt{minted} (\verb|listoflistings|).

\paragraph{Glossaries.}
To print the glossaries-related lists, uncomment:
\begin{verbatim}
% \ntaddlistof{listsofglossaries}
\end{verbatim}

\section{Institutional Presets (\texorpdfstring{\texttt{9\_*.tex}}{9\_*.tex})}
\label{sec:cfg-schools}

The files \texttt{9\_*.tex} specialize cover metadata, degree names, and School-specific requirements. Select the preset with \verb|\ntsetup{school=<id>}| in \texttt{1\_novathesis.tex}. Below are the salient controls present in the shipped presets.

\subsection*{NOVA / FCT (\texttt{9\_nova\_fct.tex})}

\begin{itemize}
  \item \textbf{Department.} Uncomment exactly one \verb|\ntdepartment(lang){…}| pair to set your Department (Portuguese and English variants are provided; list updated as of 2024--09--08).
  \item \textbf{Degree name.} A comprehensive catalogue of \verb|\ntdegreename(pt){…}| and \verb|\ntdegreename(en){…}| entries is provided for both PhD and MSc programmes. Uncomment the pair corresponding to your degree.
  \item \textbf{ADC report (DI).} Optional company fields for DI/ADC reports:
\begin{verbatim}
% \ntsetup{nova/fct/company/logo=google-logo}
% \ntsetup{nova/fct/company/name=Google}
\end{verbatim}
\end{itemize}

\subsection*{NOVA / IMS (\texttt{9\_nova\_ims.tex})}

\begin{itemize}
  \item \textbf{Sustainable Development Goals (SDGs).} The file lists SDG labels (EN/PT). Add the SDGs relevant to your work to the internal list as instructed in the comments (keep labels consistent).
\end{itemize}

\subsection*{ULisboa / IST (\texttt{9\_ulisboa\_ist.tex})}

\begin{itemize}
  \item \textbf{Final classification.} Set the final grade for the statement if required:
\begin{verbatim}
\ntsetup{classification=Aprovado com …}
\end{verbatim}
  \item \textbf{Funding.} Optionally list funding agencies:
\begin{verbatim}
% \ntsetup{funding={{FCT, projeto #1234 1234 1234},{UE, projeto #9876 ABCD}}}
\end{verbatim}
\end{itemize}

\subsection*{ULisboa / ISEG (\texttt{9\_ulisboa\_iseg.tex})}

\begin{itemize}
  \item \textbf{Funding.} Same mechanism as above (uncomment and edit).
\end{itemize}

\subsection*{ULisboa / FMV (\texttt{9\_ulisboa\_fmv.tex})}

\begin{itemize}
  \item \textbf{Exam date.} Uncomment and set the year if mandated.
  \item \textbf{Embargo.} Period \verb|I| (Immediate), \verb|6|, or \verb|12| months; optional justification text.
\begin{verbatim}
% \ntsetup{embargo/period=6}
% \ntsetup[embargo/justification]{Motivo do embargo temporário.}
\end{verbatim}
  \item \textbf{Reproduction rights.} \verb|I| (integral), \verb|P| (partial), \verb|N| (none).
\end{itemize}

\subsection*{UMinho (all) (\texttt{9\_uminho.tex})}

\begin{itemize}
  \item \textbf{Creative Commons modifier.} Select a CC variant (default is \texttt{by-nc-sa}); uncomment to override:
\begin{verbatim}
% \ntsetup{copyrightmodifier=by-nc-sa}
\end{verbatim}
  \item \textbf{Covers.} Skip the verso pages of covers for certain doctypes:
\begin{verbatim}
% \ntsetup{skipblankcovers=true}
\end{verbatim}
  \item \textbf{Integrity statement.} Optionally remove the signature line:
\begin{verbatim}
% \ntsetup{signatureline=false}
\end{verbatim}
\end{itemize}

\subsection*{UPorto / FC (\texttt{9\_uporto\_fc.tex})}

\begin{itemize}
  \item \textbf{Final classification and funding.} Same mechanisms as IST; uncomment and edit if applicable.
\end{itemize}

\section{Fonts and Engine Selection}
\label{sec:cfg-fonts}

If you choose a font style that relies on system/OpenType fonts (\texttt{arial}, \texttt{calibri}, \texttt{newsgott}, \texttt{kieranhealy}, \texttt{futura}), compile with XeLaTeX or LuaLaTeX:
\begin{verbatim}
make xe   % or: latexmk -xelatex template.tex
make lua  % or: latexmk -lualatex template.tex
\end{verbatim}
Classical LaTeX font packages (\texttt{bookman}, \texttt{erewhon}, \texttt{libertine}, \texttt{newpx}, \texttt{opensans}, \texttt{scholax}) work with pdfLaTeX as well.

\section{Abstracts and Multilingual Documents}
\label{sec:cfg-languages}

Abstract files are declared per language in \texttt{4\_files.tex}. The language order is controlled by \texttt{abstractorder} (Section~\ref{sec:cfg-main}). Additional per-language string overrides can be placed in \texttt{5\_packages.tex} using \verb|\ntlangsetup|.  
Language definitions and default strings reside under \texttt{\gls{novathesis}Files/Strings/} (EN, PT, ES, FR, DE, IT, GR, UK).

\section{Lists, Indices, and Glossaries}
\label{sec:cfg-lists}

Adjust which lists appear via \texttt{6\_list\_of.tex}. The template auto-detects algorithms/listings packages and adds corresponding lists when present. To generate glossaries/acronyms/symbols, ensure the relevant packages are loaded in \texttt{5\_packages.tex} and add \verb|listsofglossaries| if desired.

\section{Best Practices and Validation}
\label{sec:cfg-bestpractices}

\begin{itemize}
  \item \textbf{Change one thing at a time.} After editing \texttt{1\_novathesis.tex}, run a clean build (\verb|make clean|) before testing typography or cover changes.
  \item \textbf{Keep institutional choices localized.} Prefer editing only the corresponding \texttt{9\_<school>.tex}.
  \item \textbf{Avoid conflicts.} Do not redefine class internals in \texttt{5\_packages.tex}; use documented keys or language overrides.
  \item \textbf{Back up configuration.} Before upgrading the template, copy your \texttt{0-Config/} and \texttt{4-Bibliography/} folders.
\end{itemize}

\section{Minimal Configuration Checklist}
\label{sec:cfg-checklist}

\begin{enumerate}
  \item Set \verb|doctype| and \verb|docstatus| in \texttt{1\_novathesis.tex}.
  \item Select \verb|school| and complete the corresponding \texttt{9\_<school>.tex} entries (Department, Degree name, SDGs, exam date/embargo if required).
  \item Configure \verb|lang|, \verb|abstractorder|, and declare abstracts in \texttt{4\_files.tex}.
  \item Choose \verb|style/font| and compile with the appropriate engine (pdfLaTeX vs. XeLaTeX/LuaLaTeX).
  \item Confirm bibliography backend/style in \texttt{2\_biblatex.tex}; ensure \texttt{biber} is available.
  \item Review lists in \texttt{6\_list\_of.tex}; add/remove as required.
  \item Remove demo packages from \texttt{5\_packages.tex} and add only what your project needs.
  \item Build with \verb|make xe| (or \verb|make pdf|/\verb|make lua|), inspect the cover, lists, and bibliography.
\end{enumerate}
