\chapter{Artificial Intelligence Disclosure Statement}
\label{chap:aidisclose}

The \novathesis tempalte supports the rendering of a \emph{Artificial Intelligence Disclosure Statement}.

The rendering is cotrolled by the \latexinline{\ntsetup{print/aidisclosure=OPTION}} option in file \verb!0-Config/1_novathesis.tex!.

Possible values for \latexinline{OPTION} are:

\medskip
\noindent
\begin{tabularx}{\textwidth}{>{\bfseries\ttfamily}lX}
  \toprule
    false & Will omit the AI Disclosure Statement.\\
    filename.tex & Will render the given file as the AI Disclosure Statement.\\
    aidisclose & Will render the \emph{standard} AI Disclosure Statement with the help of the \href{https://github.com/joaomlourenco/aidisclose}{\texttt{aidisclose} package}.\\
  \bottomrule
\end{tabularx}


\section{Deactivating the AI Disclosure Statement}

In file \verb!0-Config/1_novathesis.tex! add the optoin
\latexinline{\ntsetup{print/aidisclosure=false}}.


\section{Using Your Own AI Disclosure Statement}

In file \verb!0-Config/1_novathesis.tex! add the optoin
\latexinline{\ntsetup{print/aidisclosure=filename.tex}}.


\section{Using the \novathesis\ Standard AI Disclosure Statement}

The \novathesis\ template uses the \href{https://github.com/joaomlourenco/aidisclose}{\texttt{aidisclose} package}~\cite{Lourenco:2025:aidisclose}, which extends the \emph{GAIDeT (Generative AI Delegation Taxonomy)}~\cite{Suchikova:2025:GAIDeT} to automate Generative AI disclosure statements and checklists.

The \texttt{aidisclose} configuration file can be found at \verb!0-Config/7_aidisclose.tex!.  Although you can edit/change this file, it is strongly recommended that you use the \href{aidisclose.org}{package's companion web} site at \url{aidisclose.org}, illustrated in \Autoref{fig:aidisclose-org}.

Fill the Author name and AI tools used, optionally add comments, leave \emph{Configuration \& Rendering} as is, and then select the appropriate checkboxes in \emph{Taxonomy of Delegated Tasks}.  Finally, press the geen button at the bottom, and replace the \novathesis aidisclose configuration file (\verb!0-Config/7_aidisclose.tex!) with the contents given in the web site.

\begin{figure}[htbp]
  \centering
    \includegraphics[height=0.9\textheight]{aidisclose-org}
  \caption{The \href{https://aidisclose.org}{\texttt{aidisclose.org} web site}}
  \label{fig:aidisclose-org}
\end{figure}

\subsection{About the AI Tags}


\begingroup
\setparaheadstyle{\normalfont\normalsize\bfseries\color{blue!80!black}}
\setlist[description]{font=\normalfont\bfseries\color{green!50!black}}

\section{Detailed Taxonomy Descriptions}
\label{app:taxonomy}

This appendix provides a detailed breakdown of the \AIDpackageName\ taxonomy.
For each category, we define the \textbf{Objective} (the high-level goal of using AI in this phase) and the \textbf{Scope} (what is generally included or excluded).

\subsection*{1. Conceptualization}
\paragraph{Objective:} To use AI as a thought partner for brainstorming, refining the research direction, and establishing the theoretical foundation before empirical work begins.
\paragraph{Scope:} Includes ideation, hypothesis formation, and feasibility checks. Excludes the actual execution of experiments or data collection.
\paragraph{Keys:}
\begin{description}
    \item[Idea generation:] Using AI to brainstorm new research topics, interdisciplinary connections, or novel angles on existing problems.
    \item[Objective refinement:] Refining vague goals into concrete, actionable research objectives.
    \item[Research questions:] Drafting and iterating on specific research questions (RQs) to ensure they are clear and answerable.
    \item[Feasibility check:] Assessing whether the proposed study is viable regarding resources, time, and data availability.
    \item[Preliminary research:] Conducting quick background checks or “pre-studies” to see if the idea has already been solved.
    \item[Simulation/Scenarios:] Using AI to conceptualize theoretical models, simulate persona responses, or design hypothetical scenarios.
\end{description}

\subsection*{2. Literature Review}
\paragraph{Objective:} To accelerate the discovery, synthesis, and organization of existing knowledge.
\paragraph{Scope:} Includes searching, summarizing, and translating papers.
Excludes the final critical argumentation (which remains the author's responsibility).
\paragraph{Keys:}
\begin{description}
    \item[Search \& Discovery:] Using AI tools (e.g., semantic search) to find relevant papers that keyword searches might miss.
    \item[Summarization:] Generating summaries or abstracts of long papers to quickly assess relevance.
    \item[Mapping:] Visualizing connections, citation networks, or thematic clusters in the literature.
    \item[Pattern recognition:] Identifying trends or recurring themes across a large corpus of text.
    \item[Gap identification:] Using AI to suggest areas where current research is lacking or contradictory.
    \item[Translation:] Translating foreign-language literature to make it accessible for the review.
\end{description}

\subsection*{3. Methodology}
\paragraph{Objective:} To assist in the structural design of the research study.
\paragraph{Scope:} Includes experimental design and instrument creation.
Excludes the physical conduct of experiments.
\paragraph{Keys:}
\begin{description}
    \item[Experimental design:] Designing the logic, control groups, and variables of the study.
    \item[Prototyping:] Creating early drafts of survey instruments, interview guides, or experimental apparatus designs.
    \item[Method selection:] Suggesting appropriate statistical methods or qualitative frameworks for the data.
\end{description}

\subsection*{4. Software Development and Automation}
\paragraph{Objective:} To facilitate the creation, optimization, and maintenance of code used in the research.
\paragraph{Scope:} Includes coding, debugging, and documentation.
\paragraph{Keys:}
\begin{description}
    \item[Code generation:] Generating boilerplate code, scripts, or functions from natural language descriptions.
    \item[Optimization:] Refactoring code for better performance or readability.
    \item[Debugging:] Identifying syntax errors or logical bugs in scripts.
    \item[Automation:] Writing scripts to automate file management, backups, or batch processing.
    \item[Algorithm design:] Assisting in the logic and mathematical formulation of algorithms.
    \item[Documentation:] Generating docstrings, comments, and README files for research software.
\end{description}

\subsection*{5. Data Management}
\paragraph{Objective:} To handle the data lifecycle from collection to reporting.
\paragraph{Scope:} Includes cleaning, analysis, and synthetic generation.
Excludes the fabrication of results (which is ethical misconduct, distinct from declared synthetic data).
\paragraph{Keys:}
\begin{description}
    \item[Collection:] Writing scrapers or using AI agents to gather public data.
    \item[Validation:] Checking data for consistency, outliers, or errors.
    \item[Cleaning:] Automating the formatting, parsing, and repair of messy datasets.
    \item[Curation:] Organizing and categorizing large datasets.
    \item[Analysis:] Suggesting or performing statistical tests and interpreting raw outputs.
    \item[Visualization:] Generating code for plots, graphs, and data dashboards.
    \item[Reporting:] Summarizing data findings in textual or tabular format.
    \item[Labeling:] Using LLMs to annotate or classify text/image datasets (zero-shot/few-shot labeling).
    \item[Synthesis:] Generating synthetic datasets to preserve privacy or augment small samples.
    \item[Anonymization:] Detecting and removing Personally Identifiable Information (PII).
    \item[Translation:] Translating textual data (e.g., survey responses) into the analysis language.
\end{description}

\subsection*{6. Visuals and Multimedia}
\paragraph{Objective:} To create or enhance non-data visual elements.
\paragraph{Scope:} Includes illustrative diagrams and image editing.
Excludes scientific data plots (covered in Data Management).
\paragraph{Keys:}
\begin{description}
    \item[Generation:] Creating conceptual images, illustrations, or diagrams from scratch.
    \item[Editing:] Enhancing, cropping, or modifying existing images (e.g., removing background).
    \item[Charts/Infographics:] Creating flowcharts, process diagrams, or high-level infographics.
\end{description}

\subsection*{7. Writing and Editing}
\paragraph{Objective:} To assist in the textual articulation of the research.
\paragraph{Scope:} Includes drafting, polishing, and translation.
Note: Authors remain accountable for accuracy.
\paragraph{Keys:}
\begin{description}
    \item[Drafting:] Generating initial text for sections based on bullet points or notes.
    \item[Polishing:] Correcting grammar, spelling, and punctuation.
    \item[Summarizing:] Creating the abstract or plain-language summary.
    \item[Conclusions:] Synthesizing the discussion into a final concluding statement.
    \item[Tone adjustment:] Rewriting text to be more formal, concise, or accessible.
    \item[Translation:] Translating the manuscript from the author's native language to the publication language.
    \item[References:] Formatting citations and bibliography styles.
    \item[Presentation:] Drafting slide decks or conference poster text.
    \item[Title generation:] Brainstorming catchy and accurate titles for the work.
\end{description}

\subsection*{8. Ethics Review}
\paragraph{Objective:} To act as a check on the ethical integrity of the work.
\paragraph{Scope:} Includes bias detection and risk assessment.
\paragraph{Keys:}
\begin{description}
    \item[Bias detection:] Scanning text or study designs for potential cultural or gender bias.
    \item[Risk assessment:] Identifying potential dual-use concerns or societal risks.
    \item[Compliance:] Checking against specific ethical guidelines or checklists.
    \item[Confidentiality:] Ensuring no private data is inadvertently leaked in the text.
\end{description}

\subsection*{9. Quality Assurance (Supervisor Role)}
\paragraph{Objective:} To use AI as a critical reviewer or “devil's advocate.”
\paragraph{Scope:} Includes simulated peer review and limitation checking.
\paragraph{Keys:}
\begin{description}
    \item[Simulated Peer Review:] Using AI as a reviewer to identify unclear sections or potential critiques.
    \item[Field Trends \& Consistency:] Checking the manuscript against current research trends and ensuring internal consistency.
    \item[Limitations:] Identifying weaknesses or limitations in the study that the authors missed.
    \item[Publication strategy:] Suggesting suitable journals or conferences and helping with the submission strategy.
    \item[Publication support:] Assisting in drafting cover letters, responses to reviewers, or other publication-related materials.
\end{description}

\endgroup
