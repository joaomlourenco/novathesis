%!TEX root = ../template.tex
%%%%%%%%%%%%%%%%%%%%%%%%%%%%%%%%%%%%%%%%%%%%%%%%%%%%%%%%%%%%%%%%%%%%
%% chapter5.tex
%% NOVA thesis document file
%%
%% Chapter with lots of dummy text
%%%%%%%%%%%%%%%%%%%%%%%%%%%%%%%%%%%%%%%%%%%%%%%%%%%%%%%%%%%%%%%%%%%%

\typeout{NT FILE chapter6.tex}%

\chapter{Document Structure}
\label{chap:document-structure}

\section{Overview}

This chapter describes the logical and physical structure of a \gls{novathesis} document.  
It explains how the thesis is partitioned into front matter, main matter, and back matter; how content files are organized on disk; and how structural elements such as tables of contents, lists, figures, tables, appendices, glossaries, and the bibliography are orchestrated.  
The intent is to provide a clear, reproducible workflow for authoring and maintaining a large academic document.

\section{Physical Layout (Project Folders)}

\gls{novathesis} separates configuration, content, and internal assets into distinct directories:

\begin{verbatim}
.
├── 0-Config/         % User-editable configuration (Chapter~\ref{chap:configuration})
├── 1-FrontMatter/    % Abstracts, acknowledgments, acronyms, dedication, quotes
├── 2-MainMatter/     % Chapters (content of the thesis)
├── 3-BackMatter/     % Appendices, annexes, optional lists (glossaries/index if used)
├── 4-Bibliography/   % Bibliographic databases (.bib files)
├── 5-Figures/        % Figures and illustrations (optional subfolders)
├── \gls{novathesis}Files/  % Internal assets (do not modify)
├── template.tex      % Main entry point (load class + orchestrate parts)
└── novathesis.cls    % Class definition (do not modify)
\end{verbatim}

Content files (\texttt{.tex}) live under the \texttt{1-}, \texttt{2-}, and \texttt{3-} directories.  
The inclusion order is defined declaratively in \texttt{0-Config/4\_files.tex}, which the class reads to assemble the final document.

\section{Logical Layout (Front/Main/Back Matter)}
\label{sec:logical-layout}

\gls{novathesis} adopts the conventional \emph{front matter} / \emph{main matter} / \emph{back matter} partitioning set by the \texttt{memoir} class, and augments it with institutional and multilingual features.

\subsection{Front Matter}
Front matter contains the cover, the title and formal pages required by the institution, and optional elements such as dedication, acknowledgments, quotes, and multilingual abstracts.  
Typical elements include:

\begin{itemize}
  \item Cover (front, verso, optional second cover; generated from metadata in \texttt{0-Config/3\_cover.tex} and the selected preset);
  \item Title and approval pages (preset-dependent);
  \item Dedication (\texttt{1-FrontMatter/dedicatory.tex}, optional);
  \item Acknowledgments (\texttt{1-FrontMatter/acknowledgements.tex}, optional);
  \item Quotes/epigraphs (\texttt{1-FrontMatter/quote.tex}, optional);
  \item Abstracts in one or more languages (files declared in \texttt{4\_files.tex});
  \item Table of Contents and lists (Figures, Tables, etc.), depending on \texttt{0-Config/6\_list\_of.tex}.
\end{itemize}

Front matter pages are typically numbered with lowercase Roman numerals, per standard academic convention, unless otherwise specified by the institutional preset.

\subsection{Main Matter}
Main matter includes the core scientific content—chapters and their subordinate sections.  
Each chapter is a separate file under \texttt{2-MainMatter/} and is declared in \texttt{0-Config/4\_files.tex} using the \texttt{chapter} kind.  
Arabic page numbering begins at the first page of the main matter.

\subsection{Back Matter}
Back matter includes appendices or annexes, glossaries, indices, and the bibliography.  
Appendices are regular chapter-level units declared with the kinds \texttt{appendix} or \texttt{annex} in \texttt{4\_files.tex}.  
The bibliography is automatically placed at the end and is built from the \texttt{.bib} files declared in \texttt{4\_files.tex}.

\section{Orchestration via \texttt{0-Config/4\_files.tex}}
\label{sec:orchestration}

The file \texttt{0-Config/4\_files.tex} serves as the canonical inventory of document parts.  
Use \verb|\ntaddfile{<kind>}[<selector>]{<name>}| to register files, where \texttt{<name>} is the basename without extension.

\subsection*{Front Matter examples}
\begin{verbatim}
% Dedication / Acknowledgments / Quote (optional)
\ntaddfile{dedicatory}{dedicatory}
\ntaddfile{acknowledgements}{acknowledgements}
\ntaddfile{quote}{quote}

% Abstracts for given languages (file names under 1-FrontMatter/)
\ntaddfile{abstract}[en]{abstract-en}
\ntaddfile{abstract}[pt]{abstract-pt}
\end{verbatim}

\subsection*{Chapters (Main Matter)}
\begin{verbatim}
% Chapters in order (files under 2-MainMatter/)
\ntaddfile{chapter}{chapter1}
\ntaddfile{chapter}{chapter2}
\ntaddfile{chapter}{chapter3}

% Restrict a chapter to specific doctypes if needed:
% \ntaddfile{chapter}[phd]{literature-review}
% \ntaddfile{chapter}[phd,phdplan]{proposal-background}
\end{verbatim}

\subsection*{Back Matter}
\begin{verbatim}
% Appendices / Annexes (files under 3-BackMatter/)
\ntaddfile{appendix}{appendixA}
\ntaddfile{appendix}{appendixB}
\ntaddfile{annex}{annex1}

% Bibliography databases (files under 4-Bibliography/)
\ntaddfile{bib}{bibliography.bib}
% \ntaddfile{bib}{secondary.bib}
\end{verbatim}

\subsection*{Covers (optional override)}
\begin{verbatim}
% User-supplied covers (override preset cover pages)
% \ntaddfile{cover}[1]{cover-front}
% \ntaddfile{cover}[N]{cover-back}
% \ntaddfile{cover}[spine]{cover-spine}

% Disable versus/second cover if your institution does not require it:
% \ntsetup{print/secondcover=false}
\end{verbatim}

\paragraph{Best practice.}  
Maintain the master sequence in \texttt{4\_files.tex} only. Avoid manual \verb|\input| in \texttt{template.tex}, which is reserved for class-level orchestration.

\section{Chapters and Sectioning}
\label{sec:chapters}

A typical chapter file under \texttt{2-MainMatter/} begins with a chapter declaration, an optional label, and freely-structured sections:

\begin{verbatim}
\chapter{Background and Related Work}
\label{chap:background}

\section{Problem Setting}
...

\subsection{Assumptions}
...

\section{Related Work}
...
\end{verbatim}

\paragraph{Numbering and depth.}  
Section numbering and ToC depth follow the class defaults and may be tuned via \texttt{memoir} options in \texttt{0-Config/0\_memoir.tex} if strictly required by the institution.

\paragraph{Labels and references.}  
Label chapters as \verb|\label{chap:<name>}| and sections as \verb|\label{sec:<name>}|.  
Use \verb|\ref| for numbered references, \verb|\nameref| for titles, and \verb|\autoref| when \texttt{hyperref} is active.  
For figures and tables, prefer \verb|\vref| (from \texttt{varioref}) if you require “on the next page” messages.

\section{Figures, Tables, Algorithms, and Listings}
\label{sec:floats}

\subsection{Figures}
\begin{verbatim}
\begin{figure}
  \centering
  \includegraphics[width=.8\linewidth]{5-Figures/method/overview}
  \caption{Method overview.}
  \label{fig:method-overview}
\end{figure}
\end{verbatim}

Store graphics under \texttt{5-Figures/} and reference using relative paths. For vector material, prefer PDF; for bitmap, PNG at print resolution.

\subsection{Tables}
\begin{verbatim}
\begin{table}
  \centering
  \caption{Hyperparameters used in all experiments.}
  \label{tab:hyperparams}
  \begin{tabular}{lcl}
    \toprule
    Name & Value & Notes\\
    \midrule
    Batch size & 64 & --- \\
    Learning rate & 1e-3 & warm-up 1k steps \\
    \bottomrule
  \end{tabular}
\end{table}
\end{verbatim}

Use \texttt{booktabs} rules (\verb|\toprule|/\verb|\midrule|/\verb|\bottomrule|) for professional tables.

\subsection{Algorithms and Listings}
\gls{novathesis} detects algorithm/listing packages and adds corresponding “List of …” entries when enabled.  
Select one algorithm package (e.g., \texttt{algorithm2e} or \texttt{algorithms}) and optionally \texttt{listings} or \texttt{minted} for code.

\begin{verbatim}
% In 5_packages.tex (example)
% \usepackage[ruled,vlined]{algorithm2e}
% \usepackage{listings} % or: \usepackage{minted}
\end{verbatim}

Then enable the lists in \texttt{0-Config/6\_list\_of.tex} (see Chapter~\ref{chap:configuration}).

\section{Tables of Contents and Lists}
\label{sec:tocs-and-lists}

The Table of Contents (ToC) is generated automatically.  
Additional lists (Figures, Tables, Algorithms, Listings, Glossaries) are controlled by \texttt{0-Config/6\_list\_of.tex}:

\begin{verbatim}
\ntaddlistof{listoffigures}
\ntaddlistof{listoftables}
% auto: listofalgorithms, lstlistoflistings or listoflistings (if packages loaded)
% glossaries bundle (if in use):
% \ntaddlistof{listsofglossaries}
\end{verbatim}

\paragraph{ToC naming and localization.}  
Override localized names via \verb|\ntlangsetup| in \texttt{5\_packages.tex} if your institution uses nonstandard terminology.

\section{Mathematics, Theorems, and Cross-Referencing}

For formal statements, you may use standard theorem packages.  
When institutional requirements demand a \emph{List of Theorems}, enable the corresponding package support (some presets leverage \texttt{coloredtheorem}).  
Cross-reference theorems with \verb|\label|/\verb|\ref|, and ensure numbering schemes are consistent across chapters.

\section{Appendices and Annexes}
\label{sec:appendices}

Appendices and annexes are separate streams of chapter-level content placed after the main matter.  
Declare them in \texttt{0-Config/4\_files.tex}:

\begin{verbatim}
\ntaddfile{appendix}{appendixA}
\ntaddfile{annex}{annex1}
\end{verbatim}

Within the appendix file, \texttt{\gls{novathesis}} switches to appendix mode automatically; you should structure content with \verb|\chapter{...}| and optional sections as needed.

\section{Glossaries, Acronyms, and Symbols}
\label{sec:glossaries}

If your thesis uses glossaries or a list of acronyms, load the relevant packages in \texttt{5\_packages.tex} and define entries in dedicated files (commonly under \texttt{1-FrontMatter/}).  
When enabled, add the glossaries list in \texttt{6\_list\_of.tex}:

\begin{verbatim}
% \ntaddlistof{listsofglossaries}
\end{verbatim}

Ensure the build toolchain includes the glossary make step when required by your chosen package (the provided Makefile and \texttt{latexmk} rules typically handle it).

\section{Bibliography Integration}
\label{sec:bibliography-structure}

Bibliographic databases are declared in \texttt{0-Config/4\_files.tex} with the \texttt{bib} kind:

\begin{verbatim}
\ntaddfile{bib}{bibliography.bib}
% \ntaddfile{bib}{bibliography_doi.bib}
\end{verbatim}

Citation style and backend settings are defined in \texttt{0-Config/2\_biblatex.tex}.  
Use \verb|\textcite|, \verb|\parencite|, and related commands consistently.  
The bibliography renders at the end of the back matter as part of the class’ standard sequence.

\section{Language and Multi-Abstract Handling}
\label{sec:language-structure}

Abstracts are per-language files declared in \texttt{4\_files.tex}.  
The order in which abstracts appear is governed by \texttt{abstractorder} in \texttt{1\_novathesis.tex} (see Chapter~\ref{chap:configuration}).  
Localization of structural strings (captions, list titles, ToC labels) is automatic based on the main language; override specific strings using \verb|\ntlangsetup|.

\section{Cover, Second Cover, and Spine}
\label{sec:cover-structure}

Covers and spine are generated from metadata in \texttt{0-Config/3\_cover.tex} and the selected institutional preset.  
If institutional rules require a second cover, it is produced automatically; disable it if not needed:

\begin{verbatim}
% 0-Config/4_files.tex (override behaviour if allowed by your School)
% \ntsetup{print/secondcover=false}
\end{verbatim}

User-supplied cover pages may be injected via the \texttt{cover} kind (Section~\ref{sec:orchestration}).  
Spine layout can be toggled and dimensioned through the keys documented in Chapter~\ref{chap:configuration}.

\section{Common Structural Variations}

\begin{itemize}
  \item \textbf{Proposal / Plan Documents} (\texttt{doctype=phdplan, mscplan}): restrict included chapters via the optional selector in \texttt{4\_files.tex} (e.g., include only background and methodology).
  \item \textbf{Articles or Reports} (\texttt{doctype=article, plain}): keep the structure minimal—front matter, a single main file, and the bibliography; remove redundant lists in \texttt{6\_list\_of.tex}.
  \item \textbf{Bilingual Documents}: declare multiple abstracts; set \texttt{lang/extra} for additional localized strings; ensure figure/table captions meet the main language rules unless your program mandates bilingual captions.
\end{itemize}

\section{Quality Assurance Checklist}

Before sharing drafts with supervisors or submitting to institutional repositories, verify:

\begin{enumerate}
  \item \textbf{Sequencing}: \texttt{4\_files.tex} lists all parts in the intended order.
  \item \textbf{Labels}: every float and structural unit has a unique, descriptive \verb|\label|.
  \item \textbf{References}: no unresolved references (\texttt{??}) or citations (\texttt{??}).
  \item \textbf{Lists}: only the required lists are enabled (Figures/Tables/Algorithms/Listings/Glossaries).
  \item \textbf{Appendices}: numbering and headings conform to institutional rules.
  \item \textbf{Language}: abstract order and ToC/labels reflect the main language and institutional policy.
  \item \textbf{Cover}: metadata, committee, sponsors/funding (if applicable) are accurate and consistent.
\end{enumerate}

\section{Summary}

\gls{novathesis} enforces a clean separation between configuration and content while adhering to academic structuring conventions.  
Define content files in \texttt{0-Config/4\_files.tex}, keep chapter files modular under \texttt{2-MainMatter/}, and allow the class to assemble front matter, lists, back matter, and bibliography in a predictable sequence.  
Following the practices outlined in this chapter ensures maintainability and compliance with institutional requirements.

