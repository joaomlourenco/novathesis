%!TEX root = ../template.tex
%%%%%%%%%%%%%%%%%%%%%%%%%%%%%%%%%%%%%%%%%%%%%%%%%%%%%%%%%%%%%%%%%%%%
%% chapter3.tex
%% NOVA thesis document file
%%
%% Chapter with a short latex tutorial and examples
%%%%%%%%%%%%%%%%%%%%%%%%%%%%%%%%%%%%%%%%%%%%%%%%%%%%%%%%%%%%%%%%%%%%

\typeout{NT FILE chapter3.tex}%

\chapter{Getting Started}
\label{chap:getting-started}

\section{Purpose of this Chapter}

This chapter provides a step-by-step guide for first-time users of the \gls{novathesis} template.  
It describes the project layout, configuration workflow, and basic compilation commands required to produce the initial document.  
By the end of this chapter, the user should be able to compile a complete sample thesis and understand where to insert their own content.

\section{Project Layout}

The \gls{novathesis} project follows a clear and modular directory structure.  
Each directory serves a specific role in the composition of the thesis.

\begin{verbatim}
_
|—— 0-Config/         Configuration files (edit here)
|—— 1-FrontMatter/    Abstracts, acknowledgments, acronyms, etc.
|—— 2-MainMatter/     Chapters and main content
|—— 3-BackMatter/     Appendices, lists, glossary, index
|—— 4-Bibliography/   Bibliographic databases (.bib files)
|—— 5-Figures/        Figures and illustrations
|—— \gls{novathesis}Files/  Internal assets (do not modify)
|—— template.tex      Main document file
|—— novathesis.cls    Class definition (do not modify)
|—— Makefile          Build automation tool
\end{verbatim}

Users should restrict their edits to the configuration and content directories.  
All other files are maintained by the template and should remain unchanged to ensure compatibility with future versions.

\section{The Role of \texttt{template.tex}}

The file \texttt{template.tex} serves as the entry point of the entire project.  
It loads the \texttt{novathesis.cls} class, applies the selected configuration, and defines the logical order of document parts:

\begin{itemize}
  \item Front matter (cover, abstract, acknowledgments);
  \item Main matter (chapters);
  \item Back matter (appendices, lists, and bibliography).
\end{itemize}

Users are not expected to modify \texttt{template.tex} beyond minor customizations such as adjusting the sequence of included chapters.  
All document-level settings are controlled through configuration files located in \texttt{0-Config/}.

\section{Configuration Overview}

The configuration subsystem is divided into modular files that correspond to different functional areas.  
Each file can be opened and edited independently.

\begin{itemize}
  \item \texttt{0-Config/1\_novathesis.tex} – Main template configuration (document type, language, school, author, title, year);
  \item \texttt{0-Config/2\_biblatex.tex} – Bibliography and citation options;
  \item \texttt{0-Config/3\_cover.tex} – Cover page customization;
  \item \texttt{0-Config/4\_files.tex}, \texttt{5\_packages.tex}, \texttt{6\_list\_of.tex} – File management and auxiliary packages;
  \item \texttt{0-Config/9\_<school>.tex} – Institutional presets (logos, formatting rules).
\end{itemize}

The most common user actions involve editing \texttt{1\_novathesis.tex} to set document metadata and institutional parameters.

\section{First-Time Setup}

To create a new thesis, follow the procedure below.

\subsection{Step 1: Copy the Template}

Create a working copy of the template directory in your preferred location.  
For example:

\begin{verbatim}
cp -r novathesis ~/Documents/MyThesis
cd ~/Documents/MyThesis
\end{verbatim}

\subsection{Step 2: Edit the Main Configuration File}

Open \texttt{0-Config/1\_novathesis.tex} in a text editor and review the default configuration.  
Update the following fields as appropriate:

\begin{verbatim}
\ntsetup{
  doctype=msc,
  school=nova/fct,
  mainlanguage=en,
  title={Deep Learning for Image Segmentation},
  author={John Doe},
  degree={Master of Science in Computer Engineering},
  department={Department of Electrical and Computer Engineering},
  year=2025
}
\end{verbatim}

Each option may be commented or adjusted according to your requirements.  
The \texttt{school} parameter activates the corresponding institutional preset.

\subsection{Step 3: Verify Language Settings}

The option \texttt{mainlanguage} determines the automatic translation of all structural elements (captions, lists, and headings).  
Supported values include:

\begin{quote}
\texttt{en}, \texttt{pt}, \texttt{es}, \texttt{fr}, \texttt{de}, \texttt{it}, \texttt{gr}, \texttt{uk}.
\end{quote}

Changing the main language automatically loads the corresponding translation file from \texttt{\gls{novathesis}Files/Strings/}.

\subsection{Step 4: Configure the Bibliography}

Open \texttt{0-Config/2\_biblatex.tex} and ensure that the bibliography processor and citation style are correctly set.  
For most users, the default configuration is appropriate:

\begin{verbatim}
\ntbibsetup{
  backend=biber,
  style=authoryear-comp,
  sorting=nyt
}
\end{verbatim}

The bibliography database is located in \texttt{4-Bibliography/bibliography.bib}.  
Add your references to this file following the standard Bib\LaTeX{} syntax.

\section{Compiling the Document}

Compilation may be performed either through the provided \texttt{Makefile} or manually using \texttt{latexmk}.  
The Makefile automates all steps required to produce the final PDF, including multiple compilation passes and bibliography generation.

\subsection{Using the Makefile}

In the project root directory, execute one of the following commands:

\begin{verbatim}
make pdf     # Build with pdfLaTeX
make xe      # Build with XeLaTeX (recommended for modern fonts)
make lua     # Build with LuaLaTeX
make clean   # Remove temporary files
make view    # Compile and open the resulting PDF
\end{verbatim}

The Makefile automatically detects the configuration from \texttt{1\_novathesis.tex}.  
Users do not need to specify the school or language explicitly.

\subsection{Manual Compilation with \texttt{latexmk}}

For environments without \texttt{make}, the document may be compiled directly as follows:

\begin{verbatim}
latexmk -xelatex template.tex
\end{verbatim}

or, for a pdf\LaTeX{} build:

\begin{verbatim}
latexmk -pdf template.tex
\end{verbatim}

To clean intermediate files, use:

\begin{verbatim}
latexmk -c
\end{verbatim}

\section{Viewing the Output}

Upon successful compilation, a file named \texttt{template.pdf} will be generated in the same directory.  
Open this file with a PDF viewer to verify that:

\begin{enumerate}
  \item The title page contains your name, title, and institution;
  \item Chapter headings and page numbering appear correctly;
  \item The sample text compiles without missing references or citations.
\end{enumerate}

\section{Editing Content Files}

After confirming that the template builds correctly, begin replacing the sample material with your own content.

\subsection{Front Matter}

Edit the files located in \texttt{1-FrontMatter/}:
\begin{itemize}
  \item \texttt{abstract.tex} – Abstract of the thesis;
  \item \texttt{acknowledgments.tex} – Acknowledgment section;
  \item \texttt{acronyms.tex} – List of acronyms and abbreviations.
\end{itemize}

\subsection{Main Matter}

Each chapter resides in \texttt{2-MainMatter/}.  
To add a new chapter, duplicate an existing file and update the chapter title:

\begin{verbatim}
\chapter{Background and Literature Review}
\label{chap:background}
\end{verbatim}

Include the new file in \texttt{template.tex} in the appropriate order.

\subsection{Back Matter}

The \texttt{3-BackMatter/} directory contains appendices, glossaries, and the bibliography.  
By default, the bibliography is automatically appended at the end of the document.  
Additional appendices may be defined using:

\begin{verbatim}
\appendix
\chapter{Supplementary Results}
\end{verbatim}

\section{Verifying Bibliography Compilation}

If citations appear as question marks (``?''), this indicates that the bibliography has not yet been processed.  
Execute the following command sequence manually to rebuild:

\begin{verbatim}
latexmk -xelatex template.tex
biber template
latexmk -xelatex template.tex
\end{verbatim}

Alternatively, running \texttt{make xe} will perform these steps automatically.

\section{Changing the Document Type}

The \texttt{doctype} parameter defines the overall structure of the document.  
Changing its value automatically adjusts formatting elements such as title page content and numbering depth.  
Supported types include:

\begin{quote}
\texttt{msc}, \texttt{phd}, \texttt{bsc}, \texttt{report}, \texttt{article}, \texttt{book}.
\end{quote}

For example:

\begin{verbatim}
\ntsetup{doctype=phd}
\end{verbatim}

\section{Customizing the Institutional Preset}

If your institution is already supported, set the \texttt{school} option accordingly.  
If not, you may define a custom preset by creating a new file \texttt{0-Config/9\_<school>.tex}.  
Use one of the existing presets as a template and modify logo placement, cover design, or metadata fields as needed.

\section{Common Issues During First Compilation}

\begin{itemize}
  \item \textbf{Fonts not found:} Ensure the required fonts for your selected theme are installed. Use XeLaTeX or LuaLaTeX for system fonts.
  \item \textbf{Bibliography missing:} Confirm that \texttt{biber} is installed and that your references are properly formatted in the \texttt{.bib} file.
  \item \textbf{Undefined control sequence:} This usually indicates a missing package. Install the full distribution (\texttt{texlive-full} or equivalent).
  \item \textbf{Images not displaying:} Check that figures are located in the \texttt{5-Figures/} directory and referenced with the correct relative path.
\end{itemize}

\section{Next Steps}

After the initial compilation and configuration are successful, the user may proceed to customize fonts, language options, and bibliography styles as described in Chapter~\ref{chap:configuration}.  
Subsequent chapters detail advanced features, including institutional extensions, font themes, and multi-language support.

At this stage, the \gls{novathesis} environment is fully operational.  
The remaining tasks involve adapting the template to the specific content and academic requirements of your degree program.
