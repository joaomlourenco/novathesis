%!TEX root=../slr.tex
\section{Conclusions}
\label{sec:conclusions}

The initial goal of this study was the assessment of the technological
status of the tomographic DOAS method for atmospheric pollutant mapping,
with the overarching objective of finding an innovative approach to the
subject.  

We have begun by identifying a set of representative electronic
libraries through a preliminary search. Then, we have constructed a
purposefully broad search phrase, which we applied to the selected
libraries. This search has rendered a total of 61 articles, of which
only 13 (around 22\%) were considered relevant and therefore further
studied. The application of the search strategy detailed in
Section~\ref{sec:methods} shows that with the great finding power of
Google's literature-oriented search engine comes an also great need for
scrutiny, since the inherent broadness of the tool results in a large
number of unwanted detections (24 exclusions vs 1 in WoS and 3 in
Scopus).

Our search has found that active tomographic DOAS is far more common
than the passive counterpart (11 out of 13 articles discussed this
method). This preference can be explained by the fact that the results
produced by this kind of system are generally superior to those obtained
by passive methods. However, passive applications are normally much less
demanding on a technical level, and are simpler to run and assemble.
Much as a result of this, we have also identified that the systems used
in the literature were not mobile or had a very low mobility level which
in turn caused that all the systems were working with low projection
numbers (as was identified in several if the selected papers). This
should be taken into account in future research on the topic.

As a final note, we would also like to point out that there seem to be
no commercially available systems for this kind of application, although
some of the articles, like the one by Stutz in 2016~\cite{Stutz2016}
detail systems which could easily be adapted to that end.


