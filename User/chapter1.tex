% 
%  chapter1.tex
%  ThesisDIFCTUNL
%  
%  Created by Vitor Duarte on 2010-03-14.
%  Copyright 2010 DI-FCT-UNL. All rights reserved.
%
\chapter{Introduction}
\label{cha:introduction}

This text will be reworked out by Vitor Duarte

% O texto da introdução, em fonte palatino 12 e com um espaçamento e meio, deve apresentar
% \begin{inparaenum}[i)]
% \item uma extensão ou introdução geral relativa ao resumo inicial;
% \item uma contextualizando o trabalho, apresentando as suas motivações;
% \item uma descrição clara do problema ou foco do trabalho; e terminando com
% \item a aproximação preconizada para a solução do problema ou do tratamento do tema focado, onde esteja claro como pensa validar o seu trabalho e quais as contribuições previstas.
% \end{inparaenum}
% 
% Os alunos podem optar por apresentar esta introdução endereçando os anteriores aspectos em sub-secções, como se exemplifica a seguir.
% 
% 
% 
% \section{Introdução geral ou Motivação}
% \label{sec:introducao}
% 
% A introdução, escrita com fonte palatino, pode ter, como referência indicativa, entre 6 e 10 páginas, usando-se um espaçamento e meio.
% 
% \section{Descrição e contexto (ou descrição do problema)}
% \label{sec:descricao}
% 
% \section{Solução apresentada (ou âmbito do trabalho)} 
% \label{sec:solucao}
% 
% \section{Principais contribuições previstas} 
% \label{sec:contribuicoes}
% 
% As principais contribuições previstas devem poder ser descritas em não mais do que uma página, podendo adoptar-se, por exemplo, um estilo de apresentação por itens, com uma pequena descrição de um parágrafo associado a cada item.
