%!TEX root = ../template.tex
%%%%%%%%%%%%%%%%%%%%%%%%%%%%%%%%%%%%%%%%%%%%%%%%%%%%%%%%%%%%%%%%%%%%
%% abstract-zhs.tex
%% NOVA thesis document file
%%
%% 摘要(中文·简体)
%% Abstract in Simplified Chinese
%%%%%%%%%%%%%%%%%%%%%%%%%%%%%%%%%%%%%%%%%%%%%%%%%%%%%%%%%%%%%%%%%%%%

\typeout{NT FILE abstract-zhs.tex}%

无论论文用何种语言撰写,通常至少需要两篇摘要:一篇使用正文语言,另一篇使用其他语言。

摘要的展示顺序可能因学院/学校要求而不同。\gls{novathesis}(\LaTeX)模板会尽量遵循各学校的默认规则;如有需要,也可以通过配置项自定义摘要的打印顺序。

建议摘要不超过一页,并尽量回答以下问题(请按你的研究领域惯例调整):

\begin{enumerate}
  \item 研究问题是什么?
  \item 为什么这个问题重要/具有挑战性?
  \item 采用了什么方法/提出了什么贡献?
  \item 得到了哪些结果(或影响/启示)?
\end{enumerate}

% 摘要关键词
\keywords{
  关键词一 \and
  关键词二 \and
  关键词三 \and
  关键词四 \and
  关键词五
}
