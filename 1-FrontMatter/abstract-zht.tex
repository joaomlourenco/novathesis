%!TEX root = ../template.tex
%%%%%%%%%%%%%%%%%%%%%%%%%%%%%%%%%%%%%%%%%%%%%%%%%%%%%%%%%%%%%%%%%%%%
%% abstract-zht.tex
%% NOVA thesis document file
%%
%% 摘要(中文·繁體)
%% Abstract in Traditional Chinese
%%%%%%%%%%%%%%%%%%%%%%%%%%%%%%%%%%%%%%%%%%%%%%%%%%%%%%%%%%%%%%%%%%%%

\typeout{NT FILE abstract-zht.tex}%

% Select a CJK font that is likely available in TeX Live (works on Linux/Windows too),
% with optional fallbacks to common system fonts.
\IfFontExistsTF{Noto Serif CJK SC}{\newfontfamily\ntzhfont{Noto Serif CJK SC}}{%
  \IfFontExistsTF{Noto Sans CJK SC}{\newfontfamily\ntzhfont{Noto Sans CJK SC}}{%
    \IfFontExistsTF{Source Han Serif SC}{\newfontfamily\ntzhfont{Source Han Serif SC}}{%
      \IfFontExistsTF{Source Han Sans SC}{\newfontfamily\ntzhfont{Source Han Sans SC}}{%
        \IfFontExistsTF{FandolSong-Regular}{\newfontfamily\ntzhfont{FandolSong-Regular}}{%
          \IfFontExistsTF{FandolSong}{\newfontfamily\ntzhfont{FandolSong}}{%
            \IfFontExistsTF{PingFang SC}{\newfontfamily\ntzhfont{PingFang SC}}{%
              \IfFontExistsTF{Heiti SC}{\newfontfamily\ntzhfont{Heiti SC}}{\newfontfamily\ntzhfont{Latin Modern Roman}}%
            }%
          }%
        }%
      }%
    }%
  }%
}
{\ntzhfont

無論論文用何種語言撰寫,通常至少需要兩篇摘要:一篇使用正文語言,另一篇使用其他語言。

摘要的顯示順序可能因學院/學校規定而不同。\gls{novathesis}(\LaTeX)樣板會盡量遵循各學校的預設規則;如有需要,也可以透過設定自訂摘要的列印順序。

建議摘要不超過一頁,並盡量回答以下問題(請依你的研究領域慣例調整):

\begin{enumerate}
  \item 研究問題是什麼?
  \item 為什麼這個問題重要/具有挑戰性?
  \item 採用了什麼方法/提出了什麼貢獻?
  \item 得到了哪些結果(或影響/啟示)?
\end{enumerate}

% 摘要關鍵詞
\keywords{
  關鍵詞一 \and
  關鍵詞二 \and
  關鍵詞三 \and
  關鍵詞四 \and
  關鍵詞五
}

}
