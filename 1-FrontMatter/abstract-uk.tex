%!TEX root = ../template.tex
%%%%%%%%%%%%%%%%%%%%%%%%%%%%%%%%%%%%%%%%%%%%%%%%%%%%%%%%%%%%%%%%%%%%
%% abstract-uk.tex
%% NOVA thesis document file
%%
%% Abstract in Ukrainian
%%%%%%%%%%%%%%%%%%%%%%%%%%%%%%%%%%%%%%%%%%%%%%%%%%%%%%%%%%%%%%%%%%%%

\begin{otherlanguage}{english}
  Translated with www.DeepL.com/Translator (free version). Proofread by Yolanda.
\end{otherlanguage}

Незалежно від того, якою мовою написана дисертація, зазвичай є щонайменше дві анотації: одна анотація тією ж мовою, що й основний текст, а інша - іншою мовою.

Порядок подання анотацій залежить від навчального закладу. Якщо у вашому навчальному закладі існують особливі правила щодо порядку розташування анотацій, шаблон \gls{novathesis} (\LaTeX) буде їх дотримуватися. В іншому випадку, як усталено, у шаблоні \gls{novathesis} на першому місці розміщується анотація \emph{тією самою, що й основний текст}, а потім анотація \emph{іншою} мовою. Наприклад, якщо дисертація написана португальською мовою, то порядок анотацій буде таким: спочатку португальська, потім англійська, а потім основний текст португальською. Якщо дисертація написана англійською мовою, то порядок анотацій буде наступним: спочатку англійська, потім португальська, а потім основний текст англійською мовою.
%
Однак цей порядок можна змінити, додавши до файлу \verb+5_packages.tex+ один з наступних рядків:
 
\begin{verbatim}
    \ntsetup{abstractorder={<LANG_1>,...,<LANG_N>}}
    \ntsetup{abstractorder={<MAIN_LANG>={<LANG_1>,...,<LANG_N>}}}
\end{verbatim}

Наприклад, для основного документа, написаного німецькою мовою, з анотаціями, написаними німецькою, англійською та італійською мовами, використовуйте (у такому порядку):

\begin{verbatim}
    \ntsetup{abstractorder={de={de,en,it}}}
\end{verbatim}

Щодо змісту, то анотації не повинні перевищувати однієї сторінки та можуть відповідати на наступні питання (важливо також дотримуватися звичайних практик вашої наукової галузі):

\begin{enumerate}
  \item У чому полягає проблема?
  \item Чому ця проблема є цікавою/складною?
  \item Який запропонований підхід/рішення/внесок?
  \item Які результати (наслідки/висновки) отримано завдяки цьому рішенню?
\end{enumerate}

% Ключові слова анотації
\keywords{
  Одне ключове слово \and
  Інше ключове слово \and
  Ще одне ключове слово \and
  Ще одне ключове слово \and
  Останнє ключове слово
}
