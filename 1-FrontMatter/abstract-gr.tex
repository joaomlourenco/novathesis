%!TEX root = ../template.tex
%%%%%%%%%%%%%%%%%%%%%%%%%%%%%%%%%%%%%%%%%%%%%%%%%%%%%%%%%%%%%%%%%%%%
%% abstract-gr.tex
%% NOVA thesis document file
%%
%% Abstract in English
%%%%%%%%%%%%%%%%%%%%%%%%%%%%%%%%%%%%%%%%%%%%%%%%%%%%%%%%%%%%%%%%%%%%

\typeout{NT FILE abstract-gr.tex}%

\textbf{Αυτή είναι μια μετάφραση “Google Translate” από την αγγλική έκδοση! Διορθώσεις και διορθώσεις είναι ευπρόσδεκτες!}

Ανεξάρτητα από τη γλώσσα στην οποία έχει γραφτεί η διατριβή, συνήθως υπάρχουν τουλάχιστον δύο περιλήψεις: μία περίληψη στην ίδια γλώσσα με το κύριο κείμενο και μία περίληψη σε κάποια άλλη γλώσσα.

Η σειρά των περιλήψεων διαφέρει ανάλογα με το πανεπιστήμιο.  
Εάν το πανεπιστήμιό σας έχει συγκεκριμένους κανονισμούς σχετικά με τη σειρά των περιλήψεων, το πρότυπο \gls{novathesis} (\LaTeX) θα τους τηρήσει.  
Διαφορετικά, ο προεπιλεγμένος κανόνας στο πρότυπο \gls{novathesis} είναι να τοποθετείται πρώτα η περίληψη στη \emph{γλώσσα του κύριου κειμένου} και στη συνέχεια η περίληψη στη \emph{δεύτερη γλώσσα}.  
Για παράδειγμα, εάν η διατριβή είναι γραμμένη στα Πορτογαλικά, η σειρά των περιλήψεων θα είναι πρώτα στα Πορτογαλικά και μετά στα Αγγλικά, ακολουθούμενη από το κύριο κείμενο στα Πορτογαλικά.  
Εάν η διατριβή είναι γραμμένη στα Αγγλικά, η σειρά θα είναι πρώτα Αγγλικά και μετά Πορτογαλικά, ακολουθούμενη από το κύριο κείμενο στα Αγγλικά.
%
Ωστόσο, αυτή η σειρά μπορεί να προσαρμοστεί προσθέτοντας μία από τις ακόλουθες εντολές στο αρχείο \verb+5_packages.tex+.

\begin{verbatim}
    \ntsetup{abstractorder={<LANG_1>,...,<LANG_N>}}
    \ntsetup{abstractorder={<MAIN_LANG>={<LANG_1>,...,<LANG_N>}}}
\end{verbatim}

Για παράδειγμα, για ένα κύριο έγγραφο γραμμένο στα Γερμανικά με περιλήψεις στα Γερμανικά, Αγγλικά και Ιταλικά (με αυτή τη σειρά) χρησιμοποιήστε:
\begin{verbatim}
    \ntsetup{abstractorder={de={de,en,it}}}
\end{verbatim}

Όσον αφορά το περιεχόμενο, οι περιλήψεις δεν πρέπει να υπερβαίνουν μία σελίδα και μπορούν να απαντούν στις ακόλουθες ερωτήσεις (είναι σημαντικό να προσαρμόζονται στις συνήθεις πρακτικές του επιστημονικού σας τομέα):

\begin{enumerate}
  \item Ποιο είναι το πρόβλημα;
  \item Γιατί είναι ενδιαφέρον/προκλητικό αυτό το πρόβλημα;
  \item Ποια είναι η προτεινόμενη προσέγγιση/λύση/συμβολή;
  \item Τι αποτελέσματα (συνέπειες/συνέπειες) από τη λύση;
\end{enumerate}

% Λέξεις-κλειδιά της περίληψης
\keywords{
  Μία λέξη-κλειδί \and
  Μια άλλη λέξη-κλειδί \and
  Μια άλλη λέξη-κλειδί \and
  Μία λέξη-κλειδί ακόμα \and
  Η τελευταία λέξη-κλειδί
}
