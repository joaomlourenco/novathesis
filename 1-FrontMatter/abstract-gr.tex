%!TEX root = ../template.tex
%%%%%%%%%%%%%%%%%%%%%%%%%%%%%%%%%%%%%%%%%%%%%%%%%%%%%%%%%%%%%%%%%%%%
%% abstract-gr.tex
%% NOVA thesis document file
%%
%% Abstract in English
%%%%%%%%%%%%%%%%%%%%%%%%%%%%%%%%%%%%%%%%%%%%%%%%%%%%%%%%%%%%%%%%%%%%

\ExplSyntaxOn
\cs_if_exist:NT \TitleOnHeader
  { \TitleOnHeader }
\ExplSyntaxOff

\typeout{NT FILE abstract-gr.tex}%

\textbf{Αυτή είναι μια μετάφραση τύπου "Google Translate" της αγγλικής έκδοσης! Διορθώσεις και βελτιώσεις είναι ευπρόσδεκτες!}

Ανεξάρτητα από τη γλώσσα στην οποία είναι γραμμένη η διατριβή, συνήθως υπάρχουν τουλάχιστον δύο περιλήψεις: μία περίληψη στην ίδια γλώσσα με το κυρίως κείμενο και μία ακόμη περίληψη σε κάποια άλλη γλώσσα.

Η σειρά των περιλήψεων διαφέρει ανάλογα με το πανεπιστήμιο.  Αν το πανεπιστήμιό σας έχει συγκεκριμένους κανονισμούς σχετικά με τη σειρά των περιλήψεων, το πρότυπο \gls{novathesis} (\LaTeX) θα τους τηρήσει.  Διαφορετικά, ο προεπιλεγμένος κανόνας στο πρότυπο \gls{novathesis} είναι να εμφανίζεται πρώτα η περίληψη στη \emph{γλώσσα του κυρίως κειμένου} και στη συνέχεια η περίληψη στην \emph{άλλη γλώσσα}.

Για παράδειγμα, αν η διατριβή είναι γραμμένη στα πορτογαλικά, η σειρά των περιλήψεων θα είναι πρώτα πορτογαλικά και έπειτα αγγλικά, ακολουθούμενα από το κυρίως κείμενο στα πορτογαλικά.  Αν η διατριβή είναι γραμμένη στα αγγλικά, η σειρά θα είναι πρώτα αγγλικά και μετά πορτογαλικά, ακολουθούμενα από το κυρίως κείμενο στα αγγλικά.

Ωστόσο, η σειρά αυτή μπορεί να προσαρμοστεί προσθέτοντας μία από τις ακόλουθες γραμμές στο αρχείο \verb+5_packages.tex+:

\begin{verbatim}
\ntsetup{abstractorder={<LANG_1>,...,<LANG_N>}}
\ntsetup{abstractorder={<MAIN_LANG>={<LANG_1>,...,<LANG_N>}}}
\end{verbatim}

Για παράδειγμα, για ένα κύριο έγγραφο γραμμένο στα γερμανικά με περιλήψεις στα γερμανικά, αγγλικά και ιταλικά (με αυτή τη σειρά), χρησιμοποιήστε:
\begin{verbatim}
\ntsetup{abstractorder={de={de,en,it}}}
\end{verbatim}

Όσον αφορά το περιεχόμενο, οι περιλήψεις δεν θα πρέπει να υπερβαίνουν τη μία σελίδα και μπορούν να απαντούν στις ακόλουθες ερωτήσεις (είναι ουσιώδες να προσαρμόζονται στις συνήθεις πρακτικές του επιστημονικού σας πεδίου):

\begin{enumerate}
\item Ποιο είναι το πρόβλημα;
\item Γιατί το πρόβλημα αυτό είναι ενδιαφέρον ή/και δύσκολο;
\item Ποια είναι η προτεινόμενη προσέγγιση / λύση / συνεισφορά;
\item Ποια αποτελέσματα (επιπτώσεις / συνέπειες) προκύπτουν από τη λύση;
\end{enumerate}

% Λέξεις-κλειδιά της περίληψης
\keywords{
Μία λέξη-κλειδί \and
Μία άλλη λέξη-κλειδί \and
Ακόμη μία λέξη-κλειδί \and
Μία επιπλέον λέξη-κλειδί \and
Η τελευταία λέξη-κλειδί
}

